\chapterspecial{Introdução}{}{Márcio M. Chaves Ferreira}\label{introduuxe7uxe3o}

\section{Forma e Conteúdo no \emph{Édipo Rei} de Sófocles}

Pode-se afirmar que uma grande parte das pessoas que hoje estão prestes
a ler ou ver encenado o \emph{Édipo Rei} já conhece ao menos os traços
gerais de seu enredo. Dos motivos que contribuíram para isso, dois
parecem estar entre os mais relevantes: desde Aristóteles (séc. \versal{IV} a.C.) 
esta tragédia de Sófocles (497/6---406 a.C.) é considerada uma
obra-prima do teatro grego, figurando não só nos currículos de diversos
cursos universitários, mas também nos do ensino médio de inúmeras
escolas do país; além disso, foi com base em elementos presentes nessa
peça que Freud deu o nome ``Complexo de Édipo'' a um dos principais
conceitos de sua teoria psicanalítica, segundo o qual todo homem está
fadado a dirigir o primeiro impulso sexual à própria mãe e o primeiro
ódio e desejo assassino contra o pai.

Mas os atenienses que viram a tragédia representada pela primeira vez,
na segunda metade do século \versal{V} a.C (428 a.C.), também estavam cientes
dos pontos mais importantes dessa tragédia. O mito de Édipo, como muitos
outros, era amplamente conhecido em várias regiões da Grécia e em tempos
ainda mais remotos que o da encenação da peça em questão. Os dois poemas
épicos de Homero (séc. \versal{VIII} a.C.), por exemplo, fazem referência a ele.

Enquanto a \emph{Ilíada} menciona apenas rapidamente o funeral de Édipo,
morto em combate (Canto 23, 677---80), a \emph{Odisseia} (Canto 11,
271---80) narra três motivos centrais do mito, que também se encontram
igualmente no drama de Sófocles: o parricídio, o incesto e o suicídio
por enforcamento da mãe/esposa de Édipo. Trata-se de uma passagem do
famoso relato de Odisseu aos feácios, no qual ele conta como foi seu
encontro no submundo com célebres heróis e heroínas já mortos, entre as
quais Jocasta, ali chamada de Epicasta:


\begin{quote}
\begin{verse}
Vi a mãe de Édipo, a bela Epicasta,\\
que em ignorância um feito grave realizou\\
ao casar com o próprio filho, que,\qb{} depois de matar seu pai,\\
casou-se com ela -- o que os deuses\qb{} logo revelaram aos homens.\\
Enquanto ele reinava na amada Tebas\qb{} sobre os cadmeus,\\
sofrendo dores por desígnios\qb{} funestos dos deuses,\\
ela partiu para o Hades,\qb{} forte guardião de suas portas,\\
após atar no teto elevado um laço pendente,\\
tomada por aflição, e a ele\qb{} deixou muitas dores\\
por vir, as quais as Erínias da mãe\qb{} vêm cumprir.
\end{verse}
\end{quote}


Já em solo ateniense e no âmbito da composição de tragédias,
aproximadamente trinta anos antes da encenação de \emph{Édipo Rei} de
Sófocles, Ésquilo (525/4---456/5 a.C.) apresenta sua trilogia
\emph{Laio}--\emph{Édipo}--\emph{Os Sete contra Tebas}, acompanhada
do drama satírico \emph{Esfinge}. Essas quatro peças foram escritas para
o festival ateniense das Grandes Dionísias, no qual três tragediógrafos
dispunham cada um de um dia para encenar um ciclo dramático com peças
que guardavam uma unidade temática entre si. Com efeito, tal unidade
pode ser observada nos títulos de Ésquilo, que representam, cada um, o
drama referente a cada geração da família real de Tebas: o drama de
Laio; o de Édipo, seu filho; e o de Etéocles e Polinices, filhos desse e
netos daquele. O chamado drama satírico (sempre uma abordagem mais leve
de episódios do mito) punha em cena a história da esfinge, que Édipo
derrotou ao chegar em Tebas. Os estudiosos apontam que uma das inovações
de Sófocles em seu \emph{Édipo Rei} teria sido romper justamente com
essa compartimentação temática, condensando em uma única peça elementos
que se encontravam dispersos nos quatro dramas esquilianos. Muito embora
o único sobrevivente desses dramas de Ésquilo seja \emph{Os Sete contra
Tebas}, citações e referências de autores posteriores oferecem evidência
da matéria tratada em \emph{Laio} e \emph{Édipo}, sendo razoável supor
que a segunda delas contivesse um enredo com conteúdo comum ao do
\emph{Édipo Rei} de Sófocles. É de se notar ainda que, como parece
evidenciar uma passagem de \emph{Os Sete contra Tebas}, mais um elemento
do mito nessa tetralogia é encontrado também em Sófocles: o
autocegamento de Édipo ao tomar consciência de seus atos.

Passando às encenações da história de Édipo em período posterior à
montagem do espetáculo de Sófocles, encontraremos um \emph{Édipo} de
Eurípides (480/79---407/6 a.C.), representado por volta de doze anos mais
tarde e do qual nos restam apenas fragmentos, e a evidência de ao menos
outras cinco montagens de peças cujo enredo gira em torno do mito desse
herói, todas atualmente perdidas.

Com a evidência de que os autores de tragédias se inseriam nessa
tradição mítica, não só entendemos melhor os motivos do já alegado
conhecimento prévio da história por parte do espectador da primeira
representação do \emph{Édipo Rei} (e somente por parte do espectador,
pois aqui não há que se falar em leitores de tragédias), como também
podemos chegar a uma conclusão mais profunda: os cidadãos atenienses não
se sentavam na plateia do teatro para assistir à representação de uma
história inédita, ou seja, totalmente nova aos seus olhos. Isso porque
nem os autores de tragédias nem os poetas que os precederam estavam
interessados em criar ficções, mas preferiam as lendas tradicionais, com
seus séculos de autoridade. Só fará sentido, portanto, dizer que o
\emph{Édipo Rei} é, como se costuma afirmar, a primeira história de
detetive da literatura ocidental, se tivermos em mente que a
investigação e o suspense envolvem apenas os personagens principais da
peça e não os seus espectadores.

Se o que atraía o interesse do espectador e servia de critério para os
juízes escolherem nesses festivais um dos tragediógrafos como vencedor
não era necessariamente a novidade do conteúdo das histórias encenadas,
ou mesmo a própria escolha desses conteúdos por parte dos autores das
peças, que elementos afinal dessas tragédias poderiam exercer esse
papel? De que características desses dramas os espectadores poderiam
extrair sua fruição? A resposta a essa questão pode ser dividida em duas
partes: uma de caráter mais geral, ligada às especificidades da tragédia
como gênero poético, e outra, mais específica, relacionada às próprias
características do \emph{Édipo Rei} de Sófocles.

\section{A forma da tragédia: poesia dramática}

Deve-se primeiramente apontar para uma característica das tragédias
gregas que até o século \versal{XVIII} era óbvia ao público teatral, mas que a
partir de então foi se tornando cada vez menos evidente: \emph{tragédia
é poesia}. ``Poesia dramática'' é justamente uma expressão de que se
valem alguns estudiosos para nomear o teatro grego, em que também se
inclui a comédia. Isso não significa apenas que o texto de uma tragédia
era composto em versos do início ao fim, mas que toda a linguagem
contida nesses versos era poética, fazendo-se notar como tal por uma
série de fatores: as expressões elevadas (que se afastavam da língua
ordinária e dos textos em prosa da época), o emprego de termos
encontrados na tradição poética anterior (principalmente nos poemas de
Homero), a atenção para a sonoridade e posição das palavras e o uso das
mais variadas figuras de linguagem.

Outra marca que contribuía para a complexidade desses espetáculos era a
presença do coro, que atuava como personagem coletivo e exercia um duplo
papel nas tragédias, ora travando diálogos diretos com os personagens,
ora se inserindo em meio à trama e cantando suas canções de caráter mais
reflexivo, sempre acompanhadas de dança -- executada pelos próprios
membros desse grupo -- e de música. Essa complexidade acima referida não
se dá apenas pelo acréscimo ao drama de um personagem coletivo, mas,
como já começamos a ver, porque esse personagem tem seus modos próprios
de exercer seu papel, modos esses que contrastam com os dos personagens
individuais. As passagens corais diferenciam-se, por exemplo, das falas
dos personagens individuais pelo fato de aquelas serem entoadas em forma
de uma canção estrófica, ao passo que essas são recitadas em versos
lineares. Esse contraste entre canção e recitação -- ao qual se poderia
apontar um paralelo, por exemplo, na ópera barroca com sua divisão em
recitativos e árias -- corresponde igualmente a uma diferença no plano do
emprego dos versos.

Enquanto o coro se expressa em seus cantos em dialeto dórico com uma
grande variedade de metros, muitos deles encontrados na poesia lírica
coral, e também num ritmo em forma de marcha, que marca sua entrada e
saída de cena, os diálogos entre os personagens individuais, seus
grandes discursos e os anúncios dos mensageiros são todos compostos em
dialeto ático, mais próximo da fala, e com um único metro, chamado
trímetro jâmbico.

Abre-se, por conseguinte, sob o ponto de vista da forma, um espetáculo
de poesia que envolvia, por um lado, a recitação encenada de versos
iguais, repetidos de forma linear pelos atores individuais e, por outro,
o canto coreografado de versos líricos assimétricos, dispostos em forma
estrófica e com marcas dialetais próprias, entoado pelo coro com
acompanhamento musical. Contudo, esse modo lírico de expressão, em forma
de canções entoadas por esse personagem coletivo, é, como já vimos de
passagem, apenas um dos dois papéis exercidos pelo coro. Ele também
dialoga com personagens individuais, valendo-se do mesmo verso regular
de que eles se valem e atuando como um personagem efetivo no enredo da
peça, e chega até em alguns momentos, que normalmente se dão próximos do
desenlace da trama, a trocar de lugar com aqueles personagens, quando
eles passam a cantar versos líricos carregados de emoção, e o coro, a
responder de modo mais contido e com versos regulares. Inclui-se, então --
na apresentação daquela estrutura composta de corais cantados -- de um
lado, e versos recitados, de outro, essa nova possibilidade de interação
entre o coro e os personagens individuais, contribuindo, agora sob o
aspecto da distribuição das falas aos personagens, para aquela
complexidade formal de que falamos acima.

Essa é a estrutura da tragédia, e a ela o tragediógrafo deverá adequar a
história que pretende pôr em cena. Nesse ponto podemos passar para o que
chamamos acima de segunda parte da resposta à questão antes formulada,
já que uma das características mais marcantes do \emph{Édipo Rei} de
Sófocles é a construção de seu enredo.

\section{Construção do enredo e ironia trágica}

Juntamente com o trabalho poético da linguagem, o gênero trágico
também busca excelência na construção do enredo dos dramas e, nesse
sentido, o \emph{Édipo Rei} é possivelmente a mais perfeita das
tragédias. A versão do mito de Édipo em que Sófocles se baseia para
escrever sua peça é resumidamente a seguinte:

Édipo nasce em Tebas. Logo após seu nascimento, seus pais, Laio e
Jocasta, ordenam que um servo da casa o abandone no monte Citéron, pois
um oráculo havia profetizado que o filho que deles nascesse mataria o
pai. O servo, com pena da criança, a entrega a um pastor, que a leva
para o palácio de Pólibo e Mérope em Corinto. Lá ele cresce convicto de
que é filho dos reis dessa cidade, até que numa festa um bêbado insinua
que Pólibo e Mérope não são seus verdadeiros pais. Desconfiado, Édipo
dirige-se então ao oráculo de Delfos para descobrir sua real filiação. O
oráculo não confirma nem desmente sua bastardia, mas o expulsa de lá,
chamando-o de assassino do pai e amante da mãe. Com medo de que essa
profecia se cumpra, Édipo decide não voltar mais para Corinto e, em seu
caminho de fuga da cidade, encontra Laio e sua comitiva, que haviam
saído de Tebas em viagem. O caminho é estreito, e Édipo não cede espaço
para a comitiva de Laio, que então o golpeia com seu bastão. Em seu
revide, Édipo mata todos, exceto um, que sobrevive sem que Édipo
perceba. Ao seguir viagem, Édipo passa pelos arredores de Tebas e se
depara com a Esfinge, que assolava os cidadãos tebanos. Ele desvenda o
enigma por ela proposto e a destrói. Como prêmio pelo feito heroico,
Édipo casa-se com Jocasta, a rainha da cidade, que, em razão da morte de
Laio, encontrava-se viúva. Passados alguns anos, uma peste terrível cai
sobre a cidade. É só nesse momento que a tragédia de Sófocles se inicia.\footnote{ Ver também no 
``Apêndice'' um resumo mais detalhado do drama conforme suas subdivisões.}

Parricídio, incesto, o suicídio de Jocasta por enforcamento, a
existência dos filhos de Édipo, a presença de Tirésias na trama, o
oráculo de Apolo, o flagelo causado pela esfinge e sua destruição por
Édipo, o autocegamento do herói, todos esses motivos do mito, que se
acham igualmente no \emph{Édipo Rei}, podem ser encontrados em fontes
anteriores ao período da produção dessa tragédia de Sófocles, de modo
que, do ponto de vista da sua matéria, ela não apresenta grandes
inovações. O aumento do número de oráculos presentes na trama (não menos
do que três) e as profecias de Tirésias, bem como a forma de abandono do
herói quando recém-nascido (com o perfuramento de seus pés e não a
simples rejeição da criança), são novidades do drama de Sófocles que não
afetam as linhas gerais da história, por mais que as passagens
divinatórias estejam intimamente ligadas às inovações na construção do
enredo.

Digna de nota é também a ocorrência da maioria daqueles acontecimentos
\emph{antes} do momento dramático em que a ação da tragédia se inicia.
Esse dado da construção do enredo aponta para uma característica
relevante e, pelo modo como é executada, inédita entre as tragédias
gregas: uma história quase toda retrospectiva, com a interrupção da
sequência cronológica dos fatos pela inserção de falas que aludem a
eventos ocorridos anteriormente.

Com a estrutura básica do enredo em mente, passemos a dois pontos
específicos de sua construção que destacam por sua relevância. São eles: 
o que se convencionou chamar de ``ironia dramática'' ou ``ironia
trágica'', e a maneira como Édipo efetua (ou como Sófocles faz que
ele efetue) as descobertas que o levam a saber quem ele realmente é e
qual a real significação dos seus atos passados. Esses dois pontos
relacionam-se com a questão do conhecimento: o segundo, por razões
óbvias, já que trata da descoberta da verdade por parte de Édipo, e o
primeiro, porque joga com a diferença entre o nível de conhecimento dos
personagens e o dos espectadores. O primeiro ponto é predominante na
primeira metade do drama, ao passo que o segundo se desenvolve mais
fortemente a partir desse momento, culminando com a fatal descoberta de
Édipo.

Diferente da figura da ironia propriamente dita, por meio da qual
dizemos o contrário do que de fato queremos dar a entender, a ironia
dramática caracteriza-se pelo descompasso entre a situação desenvolvida
no drama e as palavras que a acompanham, as quais podem ser
compreendidas de determinado modo pela plateia, mas não pelas
personagens; ou seja, o contraste não se dá entre o que se diz e o que
se quer dizer, mas entre as diferenças de percepção do significado do
que foi dito por um personagem: ele próprio, em sua ignorância, não vê
nenhuma ambiguidade em sua fala, mas a plateia, com seu conhecimento
privilegiado do mito, percebe que suas palavras significam algo além do
que ele quis dizer. Dessa forma, o conhecimento prévio da história pelo
espectador torna-se um elemento levado em conta pelo poeta para gerar
esse tipo específico de efeito irônico.

Por exemplo, quando Creonte diz a Édipo que Laio era, antes dele, o
senhor daquela terra, e ele responde que sabe disso por ter ouvido
alguém falar, pois jamais o viu, os membros da plateia que conhecem a
história de Édipo percebem de pronto a inconsistência do que foi dito.
Sim, Édipo já o viu; ele só não sabe disso! Ironia semelhante ocorre
quando, em sua busca pelo assassino de Laio, Édipo afirma: 

\begin{quote}
não dispersarei a polução em favor de amigos distantes, mas em meu próprio
benefício (...) quem quer que tenha sido o matador daquele, pode
também me querer matar com mão semelhante (vv. 137---40). 
\end{quote}

Ora, os amigos que ele estaria favorecendo não são nem um pouco distantes, mas os mais
próximos, e as mãos a que ele se refere não são outras senão as dele
mesmo. A figura também se repete na passagem em que Édipo dispõe aos
cidadãos uma série de procedimentos para localizar o assassino de Laio e
lança pesadas imprecações contra aqueles que se aproximarem desse
matador e o acolherem. A passagem como um todo, por força de seu
significado mais geral, enquadra-se obviamente nesse tipo de ironia, mas
há dois momentos dessa fala que chamam especial atenção. O primeiro
deles, quando Édipo assevera que pronunciará seu discurso como alguém
que é ``um estranho ao que se fez'', alguém que passou a pertencer à
cidade ``como um cidadão tardio'' (vv. 219---22). Bem, mesmo que não o
saiba, Édipo não é ``um estranho ao que se fez'', mas sim o responsável
pelo que se fez (pela morte de Laio). Ele tampouco é um ``cidadão
tardio'' de Tebas, mas sim seu mais legítimo membro, nascido no seio da
família real. O segundo momento se dá quando ele diz que travará esta
batalha, caracterizada pela busca do autor do homicídio contra Laio,
como se -- e aqui está o problema, nesse ``como se'' -- em favor do seu
próprio pai (vv. 264---7). Um último exemplo poderia ser tomado da
primeira fala de Tirésias na peça. Ao ser convocado para auxiliar na
descoberta dos assassinos de Laio, os quais o oráculo de Delfos
determinou que fossem mortos ou banidos da cidade, o adivinho inicia sua
fala com palavras que parecem aplicar-se a ele mesmo, mas que em seu
sentido mais profundo atingem em cheio o próprio Édipo: ``Ah! Como é
terrível o saber quando não traz proveito a quem sabe!'' (vv. 316---7).
Aqui o tema do conhecimento de novo vem à tona pela boca do adivinho,
que, na ambiguidade de sua fala, põe em movimento esses três saberes: o
dele, o da plateia e o de Édipo.

A ironia dramática prevalece na primeira metade da tragédia porque é
justamente nessa parte do drama que Édipo se mostra mais confiante em
sua inocência. E ele tinha (ou melhor, Sófocles o faz ter) boas razões
para isso. Édipo amparava-se tanto no testemunho do único sobrevivente
da matança, o qual afirmou terem Laio e todos os demais sido mortos por
mais de um assassino, quanto na sua própria suposição de que havia um
conluio entre Tirésias e Creonte para usurpar-lhe o trono, suposição
essa, aliás, também razoável. Afinal, diante dos males sofridos pela
cidade, por que o adivinho se mantivera o tempo todo em silêncio sobre o
assassino de Laio e só agora vinha acusá-lo, manifestando-se apenas
depois de ter sido convocado por sugestão de Creonte? Ora, é a confiança
de Édipo, baseada em tais convicções, que o leva a se expressar, nessa
parte da peça, com aquela linguagem da qual alguns exemplos foram
fornecidos acima.

Na segunda parte do \emph{Édipo Rei}, o conhecimento enquanto
processo de busca inexorável dos fatos torna-se o elemento central do
drama. Édipo termina por descobrir quem é o assassino de Laio e quem ele
mesmo é por meio de três descobertas encadeadas, que dão à ação uma
velocidade crescente.

Tentando aplacar a ira de Édipo contra Tirésias e mostrar-lhe a
inverdade das incriminações do adivinho (vv. 707 ss.), Jocasta menciona
o suposto lugar do assassinato de Laio, um caminho de três vias que
Édipo reconhece como o lugar onde ele mesmo, ao fugir de Corinto, matara
um senhor e sua comitiva de quatro homens. De imediato, o herói passa 
a ter fortes suspeitas sobre si mesmo, mas ainda é preciso confirmar sua
culpa com a única testemunha ocular do fato, o servo sobrevivente.

Logo em seguida, em razão da inesperada visita de um mensageiro
coríntio, dá-se uma segunda descoberta, que guarda certa analogia com a
primeira. Quando o mensageiro anuncia que Pólibo havia morrido e que
Édipo deveria assumir o trono de Corinto, o herói conta por que
deixara aquela cidade: segundo o oráculo de Delfos, ele mataria o pai e
dormiria com a mãe. Mesmo estando livre de cometer o primeiro crime, Édipo
ainda hesita em voltar, pois resta a possibilidade de cometer o segundo.
Tentando tranquilizá-lo, o mensageiro então diz que não há razão para
temer o retorno a Corinto, afirmando que nem Pólibo nem Mérope seriam
seus pais verdadeiros, pois ele próprio entregara Édipo aos reis, após
recebê-lo ainda bebê de um servo tebano que o encontrara abandonado.
Para provar que diz a verdade, o mensageiro demonstra conhecer as marcas
que Édipo carrega nos pés desde tenra infância (Sófocles, aliás, explora
o trocadilho com o nome ``Édipo'', que quer dizer ``pés inchados'' em
grego). Dessa forma, assim como as palavras de Jocasta, com a melhor das
intenções, tiveram o efeito contrário de confirmar as previsões de
Tirésias, em vez de negá-las, as palavras do mensageiro, também bem
intencionadas, em vez de negarem a mensagem do oráculo, encaminham Édipo
para a busca fatal de sua origem e o põem a um passo de ver também
cumprida aquela mensagem.

Resta apenas a última peça do quebra-cabeça para que a descoberta final
de Édipo possa mostrar a verdade dos fatos e realizar por completo a
reviravolta da ação, através da qual o herói deixa de aparecer como um
estrangeiro que é o homem mais virtuoso de Tebas ao revelar-se como seu
mais pecaminoso rebento. Isso ocorre com a chegada em cena do antigo
servo da casa de Laio, o único sobrevivente da comitiva. Do ponto de
vista dramático, é possível observar aqui um paralelismo decisivo entre
dois personagens da trama: o mensageiro de Corinto foi também o receptor
da criança abandonada, e o servo, antes convocado como sobrevivente e
única testemunha da matança de Laio e seus acompanhantes, revela-se
então como a pessoa que entregou o filho de Laio e Jocasta ao
mensageiro. Por meio do interrogatório desse homem, Édipo torna-se
ciente de tudo, e é de novo com o tema do conhecimento (ou do
reconhecimento) que voltamos a Aristóteles, para quem essa peça é uma
espécie de modelo de tragédia.

\section{Aristóteles e a excelência dramática do \emph{Édipo Rei}}

Na \emph{Poética}, Aristóteles define dois elementos que considera
constitutivos da estrutura narrativa da tragédia: o ``reconhecimento''
(\emph{anagnórisis}), que é ``a mudança da ignorância para o
conhecimento'' (1452a 28---30) e a ``peripécia'' (\emph{peripéteia}), que
é ``a mudança para a direção oposta dos eventos'' (1452a 24---25). Pois
bem, o \emph{Édipo Rei} é considerado por Aristóteles justamente o drama
que apresenta a mais bela forma de reconhecimento, ou seja, ``a que
ocorre juntamente com a peripécia'' (1452a 31---33) ou, nos termos que
usamos acima, a descoberta final de Édipo combinada com a reviravolta da
ação contra toda a expectativa do personagem. O enredo dessa tragédia é
ainda louvado por Aristóteles pelo fato de o reconhecimento que nela se
opera ser o melhor de todos, já que ``deriva dos próprios eventos,
quando a surpresa resulta de modo natural'' (1455a 16---17) e não de
sinais preparados artificialmente. Elogio semelhante também é feito ao
modo como Sófocles faz surgir na peça o ``terror'' (\emph{phóbos}) e a
``piedade'' (\emph{éleos}), emoções que por definição devem ser
suscitadas pelas tragédias: ``O enredo deve estar composto de tal modo
que, mesmo sem a visão, aquele que ouve os eventos ocorridos sinta temor
e também piedade com o que veio a acontecer, como alguém se sentiria ao
ouvir o enredo do Édipo'' (1453b 3---7). Como se vê, os três argumentos
ligam-se todos eles à construção do enredo dessa peça, pois o mais
importante, segundo o filósofo, é a estrutura dos eventos ou ``a
composição dos fatos, porque a tragédia não é imitação de homens, mas de
ação e vida'' (1450a 15---16).

A excelência dramatúrgica de Sófocles relaciona-se no \emph{Édipo Rei},
sobretudo, a esse modo de estruturação dos fatos da vida do herói
tebano. É o ritmo impresso pelo poeta ao processo de descoberta que
confere a essa tragédia seu poder único, que tanto nos fascina. Muito
desse efeito trágico se deve a uma atmosfera de desconhecimento que
permeia toda a peça e que não deixa de ser dramaticamente também
irônica: o que parece trazer a salvação atrai de fato a ruína. Já no
início do drama, o sacerdote dirige-se a Édipo como a um salvador,
quando na verdade ele é a própria causa da peste que assola a cidade;
Édipo foge da sua suposta cidade natal, para que as profecias do oráculo
não se cumpram, e chega ao lugar exato em que elas se cumprirão; as
intervenções de Jocasta e do mensageiro de Corinto, em seu intento de
aliviar a dor do rei, apenas a agravam; e Édipo, com seu conhecimento
privilegiado, que o fez desvendar o enigma da esfinge, no seu ímpeto de
tudo saber, mesmo com Jocasta o advertindo de que não levasse sua busca
adiante, vai até o fim e chega à descoberta que significará sua completa
destruição.



\chapter{Édipo depois de Édipo}


Entre as inúmeras outras representações desse mito compostas por muitos
outros autores, citamos aquelas que nos parecem mais relevantes,
apontando sucintamente alguns de seus traços principais. No âmbito da
tragédia latina, o \emph{Édipo} de Sêneca (séc. \versal{I} d.C.)
apresenta, dentro da visão estoica, um personagem principal mais passivo
do que o Édipo de Sófocles e, desde o início da peça, altamente temeroso
e desconfiado de que ele mesmo possa ser a causa da peste que assola a
cidade. Uma maior ênfase é dada ao papel de Jocasta e, portanto, ao
incesto praticado por ela e seu filho. A aparição do fantasma de Laio é
também uma inovação de Sêneca. É justamente esse \emph{Édipo} de Sêneca,
mais do que o de Sófocles, a principal fonte para a tragédia de mesmo
título de Corneille, que, no entanto, a conforma ao espírito do
teatro clássico francês do séc. \versal{XVII}. Assim, o Coro é omitido, e um
enredo secundário de amor e intriga, que expõe pretensões públicas e
privadas conflitantes, domina os primeiros atos da peça. A própria
estrutura em cinco atos é uma inovação em relação a Sófocles e Sêneca,
assim como a presença da personagem de Teseu.

No \emph{Édipo} de Voltaire, cuja estreia se deu em 1718,
nota-se uma volta da influência de Sófocles e o retorno do Coro. Como em
Corneille, a descoberta do regicídio ocorre apenas no quarto ato e a
descoberta do parentesco de Édipo no quinto. A estrutura em atos,
portanto, se mantém, e, do mesmo modo, um enredo secundário envolvendo o
tema do amor. Notável aqui é, em vez de Teseu, a inclusão de Filoctetes.
Tanto a obstinada perseguição da verdade feita por Édipo na cena em que
interroga o antigo servo de Laio quanto os ataques aos oráculos dos
deuses podem ser vistos como marcas desse \emph{Édipo} no solo do
iluminismo francês.

Já na primeira metade do séc. \versal{XX}, André Gide escreve, em 1932,
uma versão contemporânea do \emph{Édipo} sem sacrificar a forma geral da
peça de Sófocles em seu desvelamento retrospectivo do passado do herói.
A estrutura do drama é agora em três atos, e suas novidades são\quad a
inclusão dos irmãos/filhos de Édipo -- Etéocles e Polinices -- e a
amplificação dos papéis de Ismene e Antígona. Édipo aqui é um homem em
conflito com a autoridade, em especial a autoridade divina, e, nesse
caso, para uma França vista por Gide como presa a um mundo dominado pela
ortodoxia cristã (especialmente a católica), a apropriação dos clássicos
da Grécia antiga representava uma alternativa a essa autoridade e uma
possibilidade de libertação. Apropriação semelhante é feita por
Jean Cocteau em \emph{A Máquina Infernal}, de 1934, mas com
maior ênfase na teoria psicanalítica que se desenvolvia à época. O
motivo freudiano da relação incestuosa entre mãe e filho é
explicitamente explorado com a representação no terceiro ato de Édipo e
Jocasta em um quarto cuja mobília era composta ao mesmo tempo por um
berço e um grande leito conjugal. Essa versão também se recusa a
permitir que a audiência esqueça as conexões do antigo mito com o mundo
do lado de fora do teatro; Creonte, por exemplo, é apresentado como um
chefe de polícia. A peça pode ser vista como uma espécie de resposta ao
\emph{Oedipus Rex} de Stravinsky, uma ópera-oratório, com texto
do próprio Cocteau traduzido para o latim pelo padre Jean Daniélou, cuja
estreia se deu em 1927. Em seu uso do latim e em sua busca pelos ideais
de ordem, razão e claridade tão caros ao neoclassicismo francês do
pós-guerra, a obra de Stravinsky se mostra próxima daquelas tradições
conservadoras católicas de que, como vimos, tanto Gide quanto Cocteau
pretendiam se afastar ao se inspirarem nos clássicos gregos. Stravinsky,
que acabara de se converter ao cristianismo ortodoxo, fez de seu Édipo
uma figura quase sacerdotal e dotou sua peça de uma linguagem lapidar e
elevada.

A partir da segunda metade do séc. \versal{XX} montagens teatrais memoráveis do
\emph{Édipo} tornam-se mais raras. É uma representação do mito fora dos
palcos e dentro da tela de cinema que merece menção\quad o \emph{Édipo Rei}
de Pasolini, filmado em 1967 na árida paisagem do norte da
África, é um clássico do cinema, que pôs em cena um Édipo mais sensível
do que cerebral.

\asterisc{}

Vimos acima que a tragédia grega é uma manifestação essencialmente
poética e expusemos em linhas gerais algumas marcas dessa
característica, sendo a mais óbvia delas a composição de uma peça toda
em versos. No entanto, a presente tradução foi produzida em sua maior
parte em prosa, e, embora quanto a esse ponto não se mantenha fiel ao
texto original, isso se justifica por dois motivos: primeiro, porque o
texto em prosa facilita o primeiro contato do leitor leigo com a obra,
além de preservar muito bem seu conteúdo; segundo, porque os textos
dramáticos escritos atualmente privilegiam a forma prosaica.

A tradução, porém, apresenta certas passagens vertidas em versos e em
tom mais elevado, buscando uma duplicidade de expressão encontrada no
drama original. Como também mostramos nesta introdução, os versos de que
se compõem as tragédias não têm a mesma natureza, já que uma parte deles
é recitada por personagens individuais em diálogos e monólogos, ao passo
que outra é entoada pelo Coro (ou pelo Coro e um personagem individual)
em cantos estróficos com marcas dialetais distintas. Assim, o que se lê
em prosa nesta tradução são basicamente os chamados trímetros jâmbicos,
versos de ritmo mais próximo ao da fala, e o que se lê em versos são os
passos do texto compostos pelo que se costuma classificar como versos
líricos ou mélicos. Para um exame um pouco mais detido sobre as partes
da tragédia (como o prólogo, o párodo, os episódios, os estásimos e o
\emph{kommós}) e seu modo de expressão na peça, ver apêndice no fim do
livro, onde consta também um resumo do drama dividido de acordo com cada
uma dessas partes. Para uma breve explicação dos nomes de deuses,
epítetos, heróis, homens e lugares ver o glossário que antecede o
apêndice.

\asterisc

Por fim, gostaria de agradecer André Malta, Fernando Schirr, Pedro
Heise, Pedro Schmidt, Tadeu Costa e Vicente de Arruda Sampaio pelas
valiosas críticas e sugestões feitas não apenas a esta introdução, mas
