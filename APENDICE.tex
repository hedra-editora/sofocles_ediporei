\part{Apêndice} 


% \chapter{Para debater}

% 1. Os oráculos desempenham um papel decisivo na constituição do enredo
% do \emph{Édipo Rei}. Três oráculos são citados na peça. Quais são eles?
% Eles mantêm alguma relação entre si? A ordem em que eles surgem na
% tragédia reflete a sequência cronológica dos momentos em que esses
% oráculos foram proferidos?

% 2. Qual é o papel de Tirésias na tragédia? Do ponto de vista dramático,
% causa alguma estranheza que ele tenha de modo tão brusco acusado Édipo
% de assassinato já no início da peça?

% 3. Vimos que o Coro não só entoa canções, que se inserem entre um
% episódio e outro da peça, mas que também interage com outros
% personagens. Levando em contra essa sua dupla função, de que modo ele
% contribui para o desenvolvimento do enredo? Como ele se comporta em
% relação a Édipo?

% 4. O único sobrevivente da matança de Laio e de sua comitiva disse que
% eles foram atacados por ladrões e não por uma única pessoa. Isso é
% obviamente importante para o desenvolvimento da ação. Mas por que ele
% teria mentido? Seria possível imaginar algum motivo verossímil para essa
% mentira?

% 5. Uma das questões mais suscitadas sobre o \emph{Édipo Rei} é a do
% papel que o destino exerce na peça. Você acha que Édipo é uma vítima do
% destino? Ou seriam as ações do personagem que atraíram o destino sobre
% ele?

% 6. O que essa tragédia tem a nos dizer sobre a natureza da verdade? Como
% ela vem a ser descoberta por cada personagem da peça? Como isso se dá em
% relação aos membros da audiência?


\chapter{Glossário}

\begingroup
\parskip3pt
\parindent0pt

\textbf{Abas}\quad Cidade a noroeste da Fócida. Lá se encontrava um
templo de Apolo.

\textbf{Anfitrite}\quad Uma das cinquenta filhas de Nereu e Dóris. Ela se
casa com Poseidon e dá à luz Tritão.

\textbf{Arcturo}\quad A estrela que se encontra atrás da constelação Ursa
Maior (\emph{Árktos}), literalmente a vigia, a guarda (\emph{oûros}) da
Ursa. Ela aparece em meados de setembro, ou seja, no início do outono no
hemisfério norte.

\textbf{Ares}\quad Filho de Zeus e Hera, deus da guerra e da carnificina,
no \emph{Édipo Rei} também representa a peste.

\textbf{Baco}\quad Outro nome para o deus Dioniso.

\textbf{Cadmo}\quad Herói mítico, fundador da cidade de Tebas, onde se
passa a peça. Os adjetivos ``cadmeu(s)'' e ``cadmeia(s)'' referem-se a
esse herói e são, portanto, sinônimos de ``tebano(s)'' e ``tebana(s)''.

\textbf{Cilene\quad} montanha situada no nordeste da Arcádia. No
\emph{Hino Homérico a Hermes} (v. 2), esse deus é chamado ``protetor de
Cilene''.

\textbf{Citéron}\quad Nome da montanha em que Édipo, quando
recém-nascido, foi abandonado.

\textbf{Delfos}\quad O famoso oráculo de Apolo situado na Fócida ao pé do
monte Parnaso. Também chamado na peça de ``umbigo central da terra'' (v.
481) ou ``umbigo intocável da terra'' (v. 899).

\textbf{Esfinge}\quad Era um monstro, com voz humana, rosto (e, em
algumas representações, os seios) de mulher, corpo, patas e cauda de
leão, e asas de ave. Ela propunha um enigma aos transeuntes e destruía
quem não o decifrasse. A pergunta era a seguinte: ``Qual o animal que de
manhã caminha com quatro patas, ao meio-dia com duas e à tarde com três,
e que, contrariamente à lei geral, é mais fraco quando tem maior número
de membros?''. Resposta: O ser humano.

\textbf{Fásis}\quad Rio Rioni, que se origina na cordilheira do Cáucaso
e desemboca no mar Negro.

\textbf{Febo}\quad Epíteto de Apolo, derivado provavelmente de
\emph{pháos}, ``luz''. ``Febo'' seria então ``o brilhante''. O templo de
Apolo é o santuário de Delfos. Ver ``Pito'' abaixo.

\textbf{Fortuna}\quad Em grego \emph{Týkhē}. O primeiro sentido dessa
palavra é, segundo o dicionário \emph{Liddell \& Scott}, ``bem que se
obtém pelo favor dos deuses, boa fortuna, sorte, sucesso; a partir desse
significado \emph{Týkhē} passa a ser deificada, como a palavra latina
\emph{Fortuna}''. Com esse sentido, ela aparece em poemas de Píndaro e
Ésquilo, mas na \emph{Teogonia} (v. 360) de Hesíodo e no \emph{Hino
Homérico a Deméter} (v. 420) ela é uma das Oceânides, filhas de Tétis e
Oceano.

\textbf{Hades}\quad Deus das regiões subterrâneas, que por metonímia dá
também seu nome a elas. Ver o canto 15 da \emph{Ilíada} (vv. 18593),
onde a partilha do universo entre Zeus (com o céu), Hades (com as
trevas) e Poseidon (com o mar) é narrada por este último.

\textbf{Ismeno}\quad Rio de Tebas. As cinzas proféticas podem ser uma
referência a um templo de Apolo, onde havia um altar composto de cinzas
de vítimas sacrificiais.

\textbf{Istro}\quad O atual rio Danúbio.

\textbf{Lícia}\quad Região da Ásia Menor. Apesar do jogo de sonoridade,
o termo Lícia, que designa o lugar onde as referidas montanhas se
situam, não guarda, como se verá abaixo, relação de significado com
Lício.

\textbf{Lício}\quad \emph{Lýkeios} em grego. Usado no verso 203 como um
epíteto de Apolo, com significado ligado ou a lobo (\emph{lýkos}), com a
imagem do deus matador de lobos, ou a luz (\emph{leukós}), retomando o
atributo da luminosidade implícito na epíteto ``Febo'' (ver mais acima).

\textbf{Lóxias}\quad Epíteto de Apolo, derivado de ``lóxos'', ``oblíquo,
inclinado'', daí ``ambíguo'', em relação aos oráculos.

\textbf{Olímpia}\quad Cidade da Élida, onde havia um templo de Zeus. Os
Jogos Olímpicos eram ali celebrados em honra desse deus.

\textbf{Olimpo}\quad Montanha na fronteira da Tessália com a Macedônia.
Nos poemas de Homero é a morada dos deuses.

\textbf{Pã}\quad Deus da Arcádia, filho de Hermes.

\textbf{Palas}\quad Epíteto da deusa Atena, comumente derivado do verbo
\emph{pállō}, ``brandir, agitar'', sendo ela, assim, ``a que brande a
lança''.

\textbf{Parnaso}\quad Montanha da Fócida. Como o oráculo de Delfos
situava-se ao pé dessa montanha, alusões a ele são feitas também por
meio de menção a ela, como ``a mensagem revelando-se do nevado Parnaso''
(v. 474).

\textbf{Peã}\quad Originalmente um deus, médico dos deuses. Depois de
Homero, o nome e o ofício foram transferidos a Apolo. ``Peã'' também
designa um tipo de canção coral a Apolo.

\textbf{Pito}\quad Antigo nome para a parte da Fócida situada ao pé do
Monte Parnaso, onde está Delfos e se situa o oráculo de Apolo. Nas
passagens em que surge na peça de Sófocles, o termo é usado para
designar o próprio oráculo de Delfos.

\textbf{Pólibo}\quad Rei de Corinto e o suposto pai de Édipo.

\textbf{Queres}\quad Filhas da Noite, as Queres são deusas da morte
associadas ora às Moiras, cujo significado literal seria ``Partes'', e
representam o destino individual, ora às Erínias, que são as vingadoras
do sangue derramado. Sófocles usa também em outras passagens o termo
``kḗr'' (assim, no singular) em seu sentido mais geral de ``ruína,
destruição''.
\endgroup


\chapter{Estrutura e resumo do \emph{Édipo Rei}}\label{resumo}

\section{Divisão aristotélica} 

As tragédias gregas, diferentemente, por exemplo, do drama elisabetano
ou do teatro clássico francês, não se dividem em atos e cenas, mas sim
em partes recitadas e partes cantadas que, segundo a classificação
encontrada no capítulo 12 da \emph{Poética} de Aristóteles, são chamadas
de prólogo, párodo, episódio, estásimo, êxodo e \emph{kommós}. Muitos
estudiosos continuam se valendo dessa terminologia. Como optamos por
usá-la na presente tradução, vejamos como Aristóteles define cada um
desses termos:

\begin{itemize}
\item \textbf{Prólogo} ``A parte inteira da tragédia antes da entrada do
Coro'', isto é, antes do párodo propriamente dito.

\item \textbf{Párodo} ``A primeira expressão do Coro completo'', o que
significa a primeira canção na peça.

\item \textbf{Episódio} ``Uma parte inteira da tragédia entre cantos corais
completos''; podendo, portanto, estar entre o párodo e o primeiro
estásimo, ou entre um estásimo e outro.

\item \textbf{Estásimo} ``Uma canção coral sem anapestos e troqueus'', ou
seja, uma parte inteiramente cantada, já que os ``anapestos'' são metros
em ritmo de marcha e servem, por conseguinte, para indicar algumas vezes
a entrada (no início do párodo) e a saída (no fim do êxodo) do Coro de
cena. Por ``troqueus'' Aristóteles parece estar se referindo ao
``tetrâmetro trocaico'', um metro típico para as cenas compostas por
diálogos recitados e não entoados em forma de canto.

\item \textbf{Êxodo} ``A parte inteira da tragédia depois da qual não há
canção do Coro'', o que quer dizer que se trata da parte final do drama,
após o último estásimo.

\item \textbf{\emph{Kommós}} ``Um lamento em que tomam parte o Coro e a
cena'', no qual se observa não só uma interação mais imediata entre o
Coro (a partir da ``orquestra'', lugar em que se mantém durante a peça) e um
ator (a partir da cena, lugar próprio dos atores), mas também sendo
possível ocorrer uma alternância entre canto e recitação. No
\emph{kommós} é até mesmo comum que os papéis se invertam, com o
personagem individual lamentando em forma de canto e o Coro respondendo
de modo recitado. Quanto a seu significado, o termo está ligado ao verbo
\emph{kóptō}, ``bater'', em alusão ao gesto ritual dos golpes no peito
nos momentos de lamentação.
\end{itemize}

Essas partes não se combinam de uma única maneira em uma tragédia. Pela
própria definição que vimos acima, torna-se claro que haverá apenas um
prólogo, um párodo e um êxodo. Além disso, o lugar que ocuparão nas
peças é previsível: os dois primeiros no início delas, e o último no
fim. No entanto, o número de estásimos, episódios e \emph{kommoí} pode
variar, sendo também possível observar essa mesma variação em relação ao
lugar do \emph{kommós} nas peças. Dessa forma, cada peça pode ser única
do ponto de vista de sua estrutura, não estando submetida a um esquema
prévio. É bom lembrar que Aristóteles escreve a sua \emph{Poética}
aproximadamente um século depois da suposta data da apresentação do
\emph{Édipo Rei} de Sófocles, de modo que sua terminologia só pode
apresentar um caráter descritivo e não prescritivo. Segue abaixo um
quadro com essas divisões aplicadas ao \emph{Édipo Rei} e, ao lado de
cada uma delas, o seu modo de expressão e os personagens que delas
participam.


\section{Jogo de cantos e recitações}

\begin{itemize}
\item \textbf{Prólogo} (1--150.)  
 {Édipo, o Sacerdote e Creonte}. Recitação sempre no mesmo metro, o ``trímetro jâmbico''. Ele será o metro regular de todos os episódios. 

\item \textbf{Párodo} (151--215.)
 {Coro}. Canto com metros variados dispostos em três pares estróficos\footnote{ A estrutura, dos cantos corais das tragédias desenvolve-se em pares estróficos, nos quais o padrão rítmico e melódico da estrofe é espelhado na antístrofe que lhe corresponde. Esses pares podem ser seguidos por um ``epodo'', que é um novo bloco de versos com nova configuração. No \emph{Édipo Rei} isso não ocorre em momento algum.}. 

\item \textbf{I episódio} (216--462.)        
	 {Édipo, Coro e Tirésias}. Recitação sempre no mesmo metro. 

\item \textbf{I estásimo} (463--511.)     
	 {Coro}. Canto com metros variados dispostos em dois pares estróficos. 

\item \textbf{II episódio} (512--648.)         
	 {Creonte, Coro, Édipo e Jocasta}. Recitação sempre no mesmo metro.  

\item \textbf{I kommós} (649--697.)   
	 {Coro, Édipo, Creonte e Jocasta}. Um canto dialogado composto por um par estrófico. 

\item \textbf{(...) II episódio} (698--862.)
	 {Jocasta, Édipo e Coro}. Recitação sempre no mesmo metro. 

\item \textbf{II estásimo} (863--910.)         
	 {Coro}. Canto com metros variados dispostos em dois pares estróficos. 

\item \textbf{III episódio} (911--1085.)       
	 {Jocasta, o Mensageiro, Coro e Édipo}. Recitação sempre no mesmo metro. 

\item \textbf{III estásimo} (1086--1109.)      
	 {Coro}. Canto com metros variados dispostos em um par estrófico. 

\item \textbf{IV episódio} (1110--1185.)        
	 {Édipo, Coro, o Mensageiro e o Servo}. Recitação sempre no mesmo metro. 

\item \textbf{IV estásimo} (1186--1222.)        
	 {Coro}. Canto com metros variados dispostos em dois pares estróficos. 

\item \textbf{Êxodo} (1223--1296.)                  
	 {Segundo Mensageiro e o Coro}. Versos recitados sempre no mesmo metro. 

\item \textbf{II kommós} (1297--1368.)         
	 {Édipo e Coro}.Um canto dialogado, composto por dois pares estróficos, iniciando-se no verso 1313. De 1297 
	 ponto, o Coro e, em seguida, Édipo, recitam, como uma espécie de introdução ao canto que seguirá, versos anapestos.

\item \textbf{(...) Êxodo} (1369--1530.)   
	 {Édipo e Creonte}. Recitação no mesmo metro do prólogo e dos episódios anteriores até o verso 1515.\footnote{
	 desse ponto até o fim da peça, é empregado outro metro, mais raro à época de Sófocles: o tetrâmetro trocaico.}  
\end{itemize}

\section{Sinopse}

Por fim, para que se tenha uma visão geral da história, tal como se
apresenta dividida nessas partes da tragédia, vejamos um resumo com o
conteúdo dramático de cada uma delas:

\begin{itemize}
\item \textbf{Prólogo} (11--50.) Édipo surge em cena como o
grande rei, acima de quem os tebanos põem apenas os deuses. A cidade é
assolada por uma peste. Creonte retorna do oráculo de Delfos, para onde
tinha sido enviado por Édipo, dizendo que Apolo ordenava que, para se
acabar com a peste, os assassinos de Laio deveriam ser mortos ou
expulsos da cidade. Édipo se compromete em aliviar as dores dos
suplicantes que lá estão em busca de seu auxílio e perseguir os
assassinos de Laio.

\item \textbf{Párodo} (151--215.) O Coro lamenta a situação
de peste e invoca os deuses.

\item \textbf{I episódio} (216--462.) Édipo lança uma
imprecação pública sobre o assassino desconhecido de Laio. Tirésias é
convocado por Édipo por sugestão de Laio. O adivinho a princípio se
recusa a falar, mas, depois de ameaçado por Édipo e até mesmo acusado de
ser o assassino de Laio, Tirésias afirma que o assassino é o próprio
Édipo.

\item \textbf{I estásimo} (463--511.) O Coro reflete sobre a
mensagem vinda de Delfos e sobre a vida do desconhecido criminoso, que
agora está sendo caçado. O Coro se recusa a acreditar na acusação sem
provas feita por Tirésias, mantendo sua confiança em Édipo.

\item \textbf{II episódio} (512--862), com o \emph{kommós,} (649--697.) 
Creonte protesta contra a suspeita de que ele mantenha um
conluio com Tirésias para acusar Édipo. Por que ele conspiraria contra
Édipo para tomar o poder real, se de fato, como cunhado do rei, ele já
tem todos os privilégios advindos daquele poder? Édipo não se mostra
convencido. Jocasta interrompe a discussão, e Creonte parte. Édipo
conta-lhe que foi acusado de ser o assassino de Laio. Ela responde que
ele não deve se preocupar com isso, já que Laio, de acordo com um
oráculo, estava fadado a ser assassinado por seu próprio filho, mas o
bebê foi abandonado nas montanhas, e Laio foi morto por ladrões em um
ponto em que três vias se encontram. Essa menção ao encontro dos três
caminhos desperta em Édipo uma primeira desconfiança. Ele pergunta-lhe a
respeito do lugar e do tempo em que Laio foi morto, e também acerca da
sua fisionomia e do tamanho de sua comitiva. Todas as respostas de
Jocasta confirmam seu receio de que ele inadvertidamente matou Laio.
Édipo conta a Jocasta toda sua história: a suspeita em Corinto quanto a
sua verdadeira ascendência, a visita a Delfos, a fuga de sua suposta
cidade natal e o encontro na Fócida, ocasião em que matou um homem com
sua comitiva. Mas ainda há uma esperança. O servo de Laio, único
sobrevivente da matança, falou de \emph{ladrões}, e não de \emph{um}
ladrão. Que este sobrevivente seja então convocado e interrogado.

\item \textbf{II estásimo} (863--910.) O Coro profere uma prece
contra a arrogância, como a do rei em relação a Creonte, e contra a
incredulidade religiosa, que eles encontram na descrença de Jocasta em
relação aos oráculos.

\item \textbf{III episódio} (911--1085.) Um mensageiro de
Corinto anuncia que Pólibo está morto e que os habitantes de lá desejam
que Édipo se torne o novo rei. Jocasta e Édipo excitam-se com a
refutação do oráculo que havia dito a Édipo que ele mataria seu pai.
Édipo, contudo, ainda teme a outra terrível predição, isto é, a união
com sua mãe. Evita, portanto, voltar a Corinto. O mensageiro, ao saber
disso, revela que Pólibo e Mérope não são os pais de Édipo. O próprio
mensageiro, quando trabalhava como pastor no Citéron a serviço de
Pólibo, recebeu Édipo recém-nascido de outro pastor e o levou para
Corinto. Quem era esse outro pastor? O mensageiro responde: ``Era um dos
da casa de Laio''. E o Coro completa: ``Não é outro senão aquele que já
foi convocado a vir'' -- ou seja, o servo de Laio, único sobrevivente da
matança de sua comitiva. Jocasta implora que Édipo não procure saber de
mais nada. Ele responde que não se importa com quão baixo seu nascimento
possa vir se mostrar. Ele irá até o fim em sua busca. Jocasta sai de
cena soltando um grito de desespero.

\item \textbf{III estásimo} (1086--1109.) Em júbilo, o Coro
prenuncia que Édipo se revelará um nativo da terra, talvez de
ascendência divina.

\item \textbf{IV episódio} (1110--1185.) O pastor tebano é
trazido. O mensageiro de Corinto diz: ``Aí está o homem que me deu a
criança''. Ponto por ponto, toda a verdade é extraída do pastor: ``O
bebê era o filho de Laio, a esposa de Laio deu-o para mim''. Édipo agora
sabe de tudo e, com um grito de dor, sai de cena.

\item \textbf{IV estásimo,} (1186--1222.) O Coro lamenta a queda
do grande rei. Com base na história de Édipo, os mortais são igualados
ao nada. Ele, que conquistara a feliz opulência, cometeu sem saber os
piores dos crimes.

\item \textbf{Êxodo} (1223--1530.) Com o segundo \emph{kommós,}
(12971--368.) Um mensageiro vindo de dentro da casa anuncia que Jocasta
se enforcou e que Édipo rasgou seus olhos. Édipo é então trazido para
fora do palácio e num lamento impetuoso implora aos membros do Coro de
anciãos tebanos que o expulsem da terra ou o matem. Creonte chega para
levá-lo para dentro da casa. Édipo obtém dele a promessa de cuidar de
suas duas jovens filhas; elas são nesse momento levadas ao seu pai, que
entende a ocasião como um último adeus, pois ele deseja ser banido da
terra. Creonte, contudo, responde que Apolo deve se pronunciar sobre
isso. Enquanto Creonte conduz Édipo para dentro, o Coro recita suas
últimas palavras: ``Nenhum mortal deve ser chamado feliz antes que
chegue o dia de sua morte.''
\end{itemize}