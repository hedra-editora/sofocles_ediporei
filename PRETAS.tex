\textbf{Sófocles} (497/6---406 a.C.), dramaturgo ateniense, foi um dos mais eminentes 
escritores de tragédias da Antiguidade, ao lado de Ésquilo e Eurípides. De família abastada, vivenciou de perto 
o período de maior desenvolvimento cultural ateniense. Foi o poeta trágico com maior 
número de vitórias em concursos dramáticos, estreando aos 28 anos nas Grandes Dionísias, com \textit{Triptólemo}, 
atuando em muitas de suas peças e, segundo consta, tornando-se o mais amado dos tragediógrafos de seu tempo. 
Testemunhos antigos atribuem-lhe cerca de 120 dramas satíricos e tragédias, sendo que apenas sete dessas últimas nos chegaram na íntegra. 


\textbf{Édipo Rei}, desde Aristóteles (séc. \versal{IV} a.C.), é considerada uma
obra-prima do teatro grego e peça modelar à escrita de poemas trágicos, 
figurando não só nos currículos de diversos
cursos universitários, mas também nos do ensino médio de inúmeras
escolas. Além disso, foi com base em elementos presentes nessa
peça que Freud deu o nome ``Complexo de Édipo'' a um dos principais
conceitos de sua teoria psicanalítica. Pode-se, assim, afirmar que uma grande 
parte das pessoas que hoje estão prestes
a ler o \textit{Édipo Rei} já conhece ao menos os traços
gerais de seu enredo, o que em nada desabona a excelência dramatúrgica de Sófocles,
sobretudo em seu modo particular de estruturação dos fatos da vida do herói
tebano, nem tampouco diminui o fascínio motivado por essa leitura, 
devido ao ritmo impresso pelo poeta ao processo de descoberta que
confere a essa tragédia seu poder único.


\textbf{Márcio M. Chaves Ferreira} é mestre e doutorando em Letras Clássicas pela Universidade 
de São Paulo (\textsc{usp}), bacharel em Filosofia pela mesma instituição e em direito pela 
Universidade Presbiteriana Mackenzie.