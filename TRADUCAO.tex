%!TEX root=./LIVRO.tex

%\newcommand{\NL}[1]{}%\marginpar{#1}}
\marginparsep4pt


\renewcommand{\versal}[1]{{\normalfont\textsc{\MakeTextLowercase{#1}}}}
\newcommand{\saida}[1]{{\nobreak\hfill\textit{\mbox{$[$#1}}}}
\newcommand{\entrada}[1]{{\noindent\textit{#1\nolinebreak[2]}}}


\newcommand{\NL}[1]{%
 \marginpar[\raggedleft\tiny #1]{\raggedright\tiny #1}}

\newcommand{\personagem}[1]{\subsection{#1}}

\part{Édipo Rei}

\chapter{Personagens}

\vspace{-1cm}
\begingroup
\parindent0pt

\versal{Édipo,} rei de Tebas

\versal{Sacerdote}

\versal{Creonte,} irmão de Jocasta

\versal{Coro de anciãos tebanos}

\versal{Tirésias,} velho adivinho cego

\versal{Jocasta,} mulher de Édipo, rainha de Tebas

\versal{Mensageiro de Corinto}

\versal{Servo de Tebas}

\versal{Segundo Mensageiro}


\section{Figurantes mudos} 

\versal{Jovens suplicantes} 

\versal{Filhas de Édipo}

\versal{Menino guia de Tirésias}

\versal{Escravos}



\section{Época da ação} 

\versal{Idade heroica} da Grécia, entre 1200 e 1000 a.C.

\section{Local da ação} 

\versal{Tebas}

\section{Primeira representação} 

\versal{Por volta} de 428 a.C., em Atenas
\endgroup





\chapter*{Édipo Rei}

\begingroup
% \parindent-1ex

% \parskip4pt

\entrada{Diante do palácio de \versal{Édipo} em Tebas. Em frente às grandes
portas centrais há um altar; outro altar menor encontra-se também
próximo de cada uma das portas laterais.}



\section{Prólogo}

\entrada{Suplicantes  homens velhos, jovens e crianças  estão sentados
nos degraus dos altares. Sobre os altares eles depositaram ramos de
oliveira envolvidos por fitas de lã. O \versal{sacerdote} de Zeus está de pé,
sozinho, olhando para as portas centrais do palácio. Elas agora se
abrem: \versal{Édipo} entra em trajes de rei.}
\bigskip

\personagem{Édipo}   Filhos, nova descendência de Cadmo antigo, que posturas são estas em que
vos sentais, coroados com ramos de suplicantes? A cidade enche-se de
incensos, enche-se de peãs\footnote{O peã era, \emph{grosso modo}, um 
canto entoado por coro masculino, dirigido a Apolo em situações de 
peste, como a que está ocorrendo em Tebas, ou em agradecimento por uma 
vitória na guerra.} e lamentos. Por reputar injusto ouvir isso de 
outros mensageiros, filhos, eu mesmo vim, eu, que sou chamado Édipo, 
célebre a todos.

Mas, como naturalmente convém que fales por estes, conta-me, ancião, de
que\NL{10} modo vos sentis - com temor ou afeição? Eu gostaria de ajudar
em tudo, pois seria um insensível se não me apiedasse diante de tal
postura de súplica.

\personagem{Sacerdote}   Édipo, senhor da minha terra, vês com que idade nos sentamos junto aos
teus altares: uns, ainda sem forças para alçar grandes voos, outros, já
pesados pela velhice. Sou sacerdote de Zeus, e estes, escolhidos dentre
os jovens não casados. A outra parte\NL{20} dos nossos, coroada por
guirlandas, senta-se na ágora, perto dos dois templos de Palas e da
cinza profética do Ismeno. Com efeito, a cidade, como tu mesmo vês, já
se agita em demasia e não mais é capaz de levantar a cabeça das
profundezas da tormenta mortal. Ela perece nos brotos frutíferos do
solo, perece nos rebanhos de bois do pasto e nos partos malogrados das
mulheres. O deus portador do fogo, peste a mais odiosa, arrojou-se sobre
a cidade e a fustiga. Por sua causa a morada cadmeia esvazia-se e o
negro Hades enriquece-se com lamentos e gemidos.

\NL{30} Eu e estas crianças sentamos então nos degraus de teu lar, não
porque te igualas aos deuses, mas porque te julgamos o primeiro dos
homens, tanto nas vicissitudes da vida quanto no trato com os numes; tu,
que antes vieste e livraste a cidade cadmeia do tributo da rude
cantora\footnote{Alusão à Esfinge (ver ``Glossário'', p.\,\pageref{glossario}).}, o qual estávamos a
pagar, e o fizeste sem saber de nada mais nem ter sido por nós
instruído. Dizem e consideram, contudo, que foi com a assistência de um
deus que ergueste nossa vida.

\NL{40} Agora, Édipo, a todos poderosíssimo, rogamos aqui todos em súplica
que descubras algum socorro para nós, quer o conheças por teres ouvido a
palavra de algum deus, quer de um homem, pois vejo que entre os
experientes são as deliberações conjuntas que sobretudo vigoram. Vai,
melhor dos mortais, reergue a cidade! Vai, acautela-te, pois agora esta
terra te chama de salvador por teu zelo pretérito. Que de\NL{50} modo
algum nos lembremos de que em teu governo fomos postos de pé e caímos
depois, mas reergue a cidade com segurança! Como nos ofereceste uma
sorte com augúrio propício, torna-te também agora igual ao que eras
antes! Assim, se de fato vais governar esta terra, tal como estás
comandando-a, é melhor comandá-la com homens do que vazia, pois uma
muralha e um barco nada são sem homens que habitem seu interior.

\personagem{Édipo}   Filhos dignos de piedade, viestes desejosos do que me é conhecido, e não
\NL{60} desconhecido, pois bem sei que todos adoeceis; mas, mesmo doentes,
não há um de vós que adoeça como eu. A vossa dor dirige-se a uma única
pessoa em si mesma, e a nenhuma outra, mas minha alma lamenta ao mesmo
tempo a cidade, a mim e a vós. Desse modo, não me tirais do sono, mas
sabei que muitas lágrimas verti e muitos caminhos trilhei em divagações
do pensamento. O único meio de cura que em meu cuidadoso\NL{70} exame
descobri, executei-o: enviei Creonte, filho de Menécio e irmão de minha
própria esposa, ao pítico templo de Febo, para que se informe sobre o
que devo fazer ou falar para proteger esta cidade. Contando o tempo
desde esse dia, já me aflijo pelo que se passa com ele,\footnote{Ou seja,
  com Creonte.} pois se ausenta além do esperado, por mais tempo que o
conveniente. Mas, quando ele chegar, nesse momento serei vil se não
fizer tudo o que o deus revelar.

\personagem{Sacerdote}   Não só tua fala é oportuna, como também o presente anúncio destes aqui
de que Creonte se aproxima.

\personagem{Édipo}   Senhor Apolo,\NL{80} tomara que ele venha com alguma sorte salvadora,
brilhante como um olhar!

\personagem{Sacerdote}   Mas, a julgar pela aparência, está contente, pois de outro modo não
viria com a cabeça coroada de loureiro fecundo.

\personagem{Édipo}   Logo saberemos, já que está na distância certa para nos ouvir. Senhor,
meu cunhado, filho de Menécio, que palavra do deus vens nos trazer?

\entrada{Entra \versal{Creonte.}}

\personagem{Creonte}   Uma boa, pois afirmo que mesmo as dificuldades, se vierem a ter um fim
correto, podem ser superadas com todo êxito.

\personagem{Édipo}   Qual é a mensagem?\NL{90} Não estou confiante nem apreensivo com teu
presente discurso.

\personagem{Creonte}   Se desejas ouvir, mesmo com estes por perto, ou ainda ir para dentro,
estou pronto para dizer.



\personagem{Édipo}   Fala a todos. Por estes aqui sofro mais do que por minha própria vida.

\personagem{Creonte}   Vou dizer o que ouvi do deus. O senhor Febo ordena-nos claramente que
mandemos embora a mácula da cidade e, como ela se nutre nesta terra, que
não a nutramos até o ponto em que se torne incurável.

\personagem{Édipo}   Por meio de qual purificação? Qual é a forma da nossa desgraça?

\personagem{Creonte}   Com o banimento,\NL{100} ou o pagamento do homicídio com novo homicídio,
visto que esse sangue atormenta a cidade.

\personagem{Édipo}   E a que homem pertenceu esse infortúnio que ele revela?

\personagem{Creonte}   Senhor, Laio era outrora o líder desta nossa terra, antes de dirigires
esta cidade.

\personagem{Édipo}   Isso eu sei por ter ouvido, pois jamais o vi.

\personagem{Creonte}   Tendo ele morrido, agora o deus com clareza nos manda castigar com a mão
os assassinos, quem quer que eles sejam.

\personagem{Édipo}   Em que lugar da terra eles estão? Onde será descoberto o rastro
indistinto desta antiga culpa?

\personagem{Creonte}   Nesta terra,\NL{110} afirmava ele. O que se busca pode ser alcançado, mas
escapa aquilo de que se descuida.

\personagem{Édipo}   Laio depara com esse assassinato em casa, ou nos campos, ou em outra
terra?

\personagem{Creonte}   Ia, segundo afirmava, como embaixador a Delfos, e não mais voltou para
casa desde que se retirou.

\personagem{Édipo}   Nenhum mensageiro nem companheiro de viagem viu algo de que se poderia
fazer uso, caso alguém o soubesse?

\personagem{Creonte}   Estão mortos, exceto um, que fugiu de medo e nada do que viu podia
contar com certeza, exceto um fato.

\personagem{Édipo}  Qual? Quando \NL{120} se compreende um fato, muitos podem ser depois
descobertos, se se começa com um pouco de zelo.

\personagem{Creonte}   Ele dizia que ladrões os encontraram e o mataram não com a força de um
só homem, mas com múltiplas mãos.

\personagem{Édipo}   Como o ladrão\footnote{Essa passagem para o singular da palavra que na
  fala de Creonte encontrava-se no plural está no texto original. Seria
  ela mais uma das ironias da peça?} chegaria a este grau de audácia, se
não tivesse havido aqui alguma negociação com dinheiro?

\personagem{Creonte}   É o que parecia. Mas, depois que Laio morreu, ninguém veio a nos
assistir contra nossos males.

\personagem{Édipo}   Quando o poder real assim desabou, que mal se pôs em vosso caminho e vos
impediu de saber tudo?

\personagem{Creonte}   A Esfinge, cantora\NL{130} variegada, induzia-nos a abandonar o invisível
e investigar o que estava a nossos pés.

\personagem{Édipo}   Mas o farei de novo visível desde o início. Febo, com todo valor, e tu,
valorosamente também, assim vos voltastes ao morto, de modo que também a
mim vereis com toda a justiça como um aliado, a vingar tanto esta terra
quanto o deus. Não é em favor de amigos distantes, mas em favor de mim
que eu mesmo dispersarei essa\NL{140} polução. Quem quer que tenha sido o
seu matador pode também querer me matar com mão semelhante. Assim, ao
socorrer aquele, beneficio a mim mesmo.

Vamos, levantai-vos, filhos, com toda pressa dos degraus e retirai estes
ramos de suplicantes; e que alguém aqui reúna o povo de Cadmo, pois tudo
farei. Com o auxílio do deus nos mostraremos ou venturosos ou
arruinados.

\personagem{Sacerdote}   Filhos, levantemo-nos, pois viemos para cá por causa disso mesmo que ele
\NL{150} está a anunciar. Que Febo, que enviou estas profecias, venha como
salvador e cesse o flagelo!
\bigskip

\entrada{O \versal{sacerdote}, as crianças, \versal{Édipo} e \versal{Creonte} saem de cena. O \versal{Coro} de
anciãos tebanos ingressa e entoa a primeira canção da peça}.


\pagebreak

\section{Párodo}


\personagem{Coro}

\entrada{1ª estrofe}


\settowidth{\versewidth}{brilhante? Tenso, agito-me com medo em meu trêmulo}
\begin{verse}[\versewidth]
Ó doce oráculo de Zeus, quem és tu, que vieste\\
da multiáurea Pito para Tebas\\
brilhante? Tenso, agito-me com medo em meu\qb trêmulo peito,\\
ó délio\footnote{``Délio'' refere-se a Delos, lugar de nascimento do
        deus Apolo.} Peã\footnote{Aqui como epíteto de Apolo (ver ``Glossário'', p.\,\pageref{glossario}),
								  um uso do termo diverso daquele apontado na nota 1.}%
					    , deus dos gritos,\\
em temor de ti! Que dever me cobrarás?\\
Um novo, ou um que retorna com a passagem\qb das estações?\\
Diz-me, ó Fala imortal, filha da Esperança dourada.\\!
\end{verse}

\entrada{1ª antístrofe}

\settowidth{\versewidth}{%
brilhante? Tenso, agito-me com medo em meu trêmulo}
\begin{verse}[\versewidth]
Chamo-te primeiro, Atena imortal, filha de Zeus,\\
e tua irmã, Ártemis,\NL{160}\\
protetora da terra, assentada em glorioso trono\qb circular da ágora,\\
e Febo, longe-atirante, ah!\\
Aparecei para mim, tríade que afasta a morte:\\
se alguma vez, quando pestes pretéritas ergueram-se\qb contra a cidade,\\
expulsastes daqui a chama da dor, vinde também agora!\\!
\end{verse} 

\pagebreak
\entrada{2ª estrofe}
\settowidth{\versewidth}{brilhante? Tenso, agito-me com medo em meu trêmulo}
%xxxxxxxxxxxxxxxxxxxxxxxxxxxxxxxxxxxxxxxxxxxxxxxxxxx
\begin{verse}[\versewidth]
Oh! Dores inúmeras suporto;\\
toda a minha tropa\\
adoece, e não há arma do pensamento\NL{170}\\
com que nos defendamos. Não prosperam\\
os rebentos da terra renomada, nem mulheres\\
se livram dos trabalhos cheios de gritos dos partos.\\
Pode-se ver cada um em cada parte, como um pássaro\qb de asas ágeis,\\
avançando com mais força do que o fogo imbatível\\
para a costa do deus crepuscular.\footnote{A referência é a Hades (ver ``Glossário'', p.\,\pageref{glossario}).}\\!
\end{verse} 

\entrada{2ª antístrofe}
\begin{verse}[\versewidth]
Com essas perdas inúmeras a cidade perece;\\ %2ª antístrofe
trazendo a morte, jaz no chão sua descendência\NL{180}\\
sem que a lamentem ou dela se apiedem,\\
e esposas e mães grisalhas\\
ao longo da margem dos altares gemem\\
de um lado e de outro em súplica contra a triste fadiga.\\
Lampeja o peã\footnote{Ver nota 1.} % apontar para linha 260 
e em concerto com ele uma\qb voz lamentosa.\\
Por isso, ó filha dourada de Zeus,\\
manda-nos o socorro de bela face.\\!
\end{verse} 
\entrada{3ª estrofe}

\settowidth{\versewidth}{brilhante? Tenso, agito-me com medo em meu trêmulo}
%xxxxxxxxxxxxxxxxxxxxxxxxxxxxxxxxxxxxxxxxxxxxxxxxxxx
\begin{verse}[\versewidth]
Possa Ares, o impetuoso,\NL{190}\\
que agora, sem o bronze dos escudos,\\
se inflama com gritos diante de mim,\\
virar as costas para esta pátria e tomar o curso reverso,\\
sob vento propício, para o quarto\\
vasto de Anfitrite\\
ou para as inóspitas ondas\\
trácias longe dos portos.\\
Se a noite algo deixa escapar,\\
vem o dia e o executa.\\
A ele, ó detentor da força\NL{200}\\
dos relâmpagos que portam fogo,\\
ó Zeus pai, destrói com teu raio!\\!
\end{verse}

\entrada{3ª antístrofe}

\settowidth{\versewidth}{brilhante? Tenso, agito-me com medo em meu trêmulo}
%xxxxxxxxxxxxxxxxxxxxxxxxxxxxxxxxxxxxxxxxxxxxxxxxxxx
\begin{verse}[\versewidth]
Lício senhor, eu desejaria\\ 
celebrar as setas indomáveis\\
vindas da corda dourada de teu arco,\\
postas à frente como defesa, e os fachos\\
de Ártemis, portadores do fogo, com os quais\\
ela atravessa as montanhas da Lícia;\\
e chamo o de áureo diadema,\\
deus epônimo desta terra,\NL{210}\\
Baco, face de vinho,\\
em companhia das mênades sob gritos de ``evoé'',\\
para que, em chamas de pinheiro\\
radiante, se aproxime\\
contra o deus entre os deuses desonrado.\footnote{O deus entre os deuses desonrado é Ares (ver ``Glossário'', p.\,\pageref{glossario}), já referido no início da terceira estrofe.}
\end{verse}


\pagebreak
\section{Primeiro episódio}

\entrada{\versal{Édipo} entra.}

\personagem{Édipo}   Pedes e, quanto ao que pedes, se quiseres ouvir minhas palavras,
aceitá-las e auxiliar no combate ao flagelo, poderás receber proteção e
alívio dos males. O que eu disser, direi não só como um estranho ao que
se contou, mas como um estranho ao que\NL{220} se fez; com efeito, eu não
seguiria por mais tempo esse rastro, se não tivesse sido de algum modo
acolhido por vós. Agora então, já que como cidadão tardio passo a
pertencer a esta cidade, proclamo a todos vós, cadmeus, isto: quem quer
que dentre vós saiba que homem matou Laio, filho de Lábdaco, a esse
ordeno que me revele tudo. Se, ao afastar a acusação da cidade, ele teme
atrair a pena de morte sobre si mesmo, pois bem, nenhuma outra
animosidade sofrerá e partirá da terra ileso. Mas, se alguém sabe\NL{230}
que outro, daqui ou de outra terra, foi o autor do crime, que não se
cale, pois pagarei recompensa à qual se somará minha gratidão. Se, por
outro lado, vos calardes, e alguém, aflito por causa de um amigo ou
mesmo por si próprio, repelir esse meu decreto, deveis ouvir de mim o
que farei em seguida. Proíbo que todos desta terra, cujo poder e trono
detenho, acolham esse homem, quem quer que ele seja, lhe dirijam
palavra, com ele compartilhem das preces ou sacrifícios aos deuses e lhe
ofereçam água\NL{240} lustral; e ordeno a todos que o afastem das casas,
visto que ele é mácula entre nós, como o oráculo pítico há pouco me
revelou. Alio-me, pois, deste modo, tanto à divindade quanto ao homem
que morreu. Rezo para que o autor do feito, quer tenha ele escapado só,
quer em companhia de outros mais, leve desgraçadamente, como um
desgraçado, uma vida infeliz. Peço, além disso, que, se ele compartilhou
em minha casa\NL{250} do fogo do lar com minha ciência, eu sofra o que há
pouco imprequei contra esses.

Ponho sobre vós o encargo de cumprir tudo isso, não só em meu favor, mas
também pelo deus e por esta terra em ruínas, assim sem frutos e
abandonada pelos deuses. Mesmo se essas circunstâncias não tivessem sido
provocadas por um deus, não seria natural que vós as deixásseis impuras
desse modo, mas, já que um homem excelente, um rei, está morto, a
buscásseis até o fim. E agora, como detenho o comando\NL{260} que ele
antes detinha, detenho leito e esposa partilhados com ele; se sua
descendência não tivesse malogrado, haveria nascimentos de filhos em
comum -- mas agora o destino se lançou sobre a sua cabeça. Por esses
motivos, travarei esta batalha como se em favor de meu pai e irei a toda
parte, buscando encontrar o autor do homicídio contra o filho de
Lábdaco, e descendente de Polidoro, e antes de Cadmo, e, há muito, de
Agenor.\footnote{Genealogia de Laio, que era filho de Lábdaco, filho de
  Polidoro, filho de Cadmo, filho de Agenor, rei da Fenícia. O que Édipo
  ainda não sabe é que essas gerações não se findaram com a morte de
  Laio.}

Aos que não fizerem isso, rogo aos deuses que não lhes concedam colheita
\NL{270} alguma da terra, nem filhos de mulheres, mas venham a se arruinar
pelo mal que ora se abate ou por um outro ainda mais odioso. Mas quanto
a vós, demais cadmeus, aos quais essas palavras agradam, que a aliada
Justiça e todos os deuses possam estar de bem convosco para sempre.

\personagem{Coro}   Como me prendeste sob uma imprecação, senhor, assim direi. Não matei nem
tenho como indicar aquele que matou. Quanto à busca, era para Febo, que
enviou essa mensagem, dizer quem afinal praticou o ato.

\personagem{Édipo} Falaste com justeza,\NL{280} mas nenhum homem teria o poder de forçar os
deuses ao que não queiram.

\personagem{Coro}   Eu poderia dizer o que me parece a segunda opção depois dessa?

\personagem{Édipo}   Se houver até mesmo uma terceira, não deixes de me contar.

\personagem{Coro}   Senhor por senhor, sei que Tirésias tem a visão mais semelhante à de
Febo. Se junto a ele isto se investigasse, senhor, seria possível
aprendê-lo com toda clareza.

\personagem{Édipo}   Mas nem mesmo isso deixamos negligenciado, pois lhe enviei, depois que
Creonte o sugeriu, dois emissários; e sua ausência há muito me espanta.

\personagem{Coro} De fato,\NL{290} o que resta são palavras antigas e sem sentido.

\personagem{Édipo}   Quais são essas? Examino qualquer relato.

\personagem{Coro}   Contou-se que ele morreu pelas mãos de alguns viajantes.

\personagem{Édipo}   Isso também eu ouvi, mas ninguém vê o autor\footnote{Essa é uma correção
  antiga ({1779}) e anônima, usada por Lloyd"-Jones e por grande parte dos
  editores para o texto dos manuscritos, que curiosamente dão: ``mas
  ninguém vê `o que viu'\,''.} do feito.


\personagem{Coro}   Mas agora, se ele tiver alguma parcela de medo, não permanecerá aqui
depois de ouvir as tuas imprecações.

\personagem{Édipo}   A quem faz isso sem medo, palavras não afugentam.

\personagem{Coro}   Mas apresenta-se aquele que há de pô-lo à prova, pois estes para cá
conduzem o divino adivinho, único dos homens que dentro de si possui a
verdade.

\entrada{\versal{Tirésias}, conduzido por um menino, entra em cena}.

\personagem{Édipo}   Tirésias,\NL{300} tu que tudo observas, o apreensível e o inefável, o que
está no céu e o que pisa a terra: mesmo que não vejas, sabes contudo com
que tipo de flagelo a cidade convive. Apenas em ti descobrimos quem dele
nos proteja e salve, senhor. Se ainda não ouviste os mensageiros, Febo
enviou a nós, que lhe enviáramos consulta, resposta segundo a qual a
libertação desta enfermidade somente poderá vir se bem compreendermos
quem foram os matadores de Laio, e os matarmos ou os expulsarmos\NL{310}
da terra como exilados. Sem então nos recusar a mensagem dos pássaros,
nem qualquer outro caminho da arte profética, se o tiveres, protege-te a
ti mesmo e à cidade, protege a mim, protege-nos de toda mácula do morto.
Estamos em tuas mãos. O mais belo trabalho de um homem é ajudar com o
que quer que tenha e possa.

\personagem{Tirésias}   Ah! Como é terrível o saber quando não traz proveito a quem sabe! Embora
tivesse pleno conhecimento disso, o perdi da lembrança; caso contrário,
não teria vindo aqui.

\personagem{Édipo}   O que há? Como chegas desanimado!

\personagem{Tirésias}   Deixa-me ir\NL{320} para casa. Se me obedeceres, suportarás com toda
facilidade o teu fardo; e eu, o meu.

\personagem{Édipo}   Ao nos privares deste anúncio, proferes palavras que não são justas nem
amigáveis a esta cidade que te nutriu.

\personagem{Tirésias}   Sim, pois vejo que teu discurso se encaminha para o que não te convém. A
fim de que então eu não padeça o mesmo mal\ldots{}

\personagem{Édipo}   Pelos deuses! Se de fato sabes, não te desvies de nós, quando todos aqui
nos prostramos em súplica a ti!

\personagem{Tirésias}   Porque vós todos não sabeis. Que eu jamais revele os meus males, para
não dizer os teus.

\personagem{Édipo}   Que dizes? Mesmo\NL{330} tendo esse conhecimento, não nos contarás? Pensas
em nos trair e em destruir a cidade?

\personagem{Tirésias}   Não causarei dor a mim, nem a ti. Por que me questionas assim em vão?
Não poderás obter essa informação de mim.

\personagem{Édipo}   Ó mais perverso dos perversos, provocarias a ira até mesmo de uma pedra!
Jamais falarás, mas assim inflexível e impraticável te mostrarás?

\personagem{Tirésias}   Repreendeste a minha disposição, mas não observaste a que habita ao
mesmo tempo contigo\footnote{Fala aparentemente ambígua de Tirésias que
  alude à disposição de ânimo de Édipo, palavra também feminina em grego
  (\emph{orgé}), valendo-se do verbo habitar, i.e., Édipo não percebe
  quem é aquela (disposição ou mulher?) que habita com ele.} e me
censuras.


\personagem{Édipo}   Quem não se enraiveceria\NL{340}  ao ouvir tais palavras, com que agora
desonras esta cidade?

\personagem{Tirésias}   Os fatos virão por si mesmos, mesmo que eu os encubra com silêncio.

\personagem{Édipo}   Mas não és tu quem me deve dizer o que virá?

\personagem{Tirésias}   Não posso expor mais nada. Quanto a isso, enfurece-te, se quiseres, com
a ira mais selvagem.

\personagem{Édipo}   Sim, tenho tanta ira que não deixarei passar nada do que compreendo!
Fica sabendo que, para mim, de fato planejaste com alguém o feito e o
fizeste, muito embora não o tenhas matado com as mãos. Se acaso
enxergasses, eu diria que esse feito foi de tua única autoria.

\personagem{Tirésias}    Verdade? Recomendo,\NL{350} pois, que permaneças fiel ao próprio decreto
que proferiste e que, a partir do dia de hoje, não te dirijas a estes
homens, nem a mim, uma vez que és o sacrílego que traz a mácula a esta
terra.

\personagem{Édipo}   Assim, despudoradamente, soltas essa declaração? Como pensas que hás de
escapar disso?

\personagem{Tirésias}   Já escapei, pois nutro uma verdade que é forte.\footnote{A partir daqui até o verso 365, temos um
  exemplo do que se chama de \emph{esticomitia}, uma passagem da
  tragédia em que os personagens recitam apenas um verso a cada vez, o
  que lhe confere um alto grau de intensidade emotiva.}

\personagem{Édipo}   Quem te informou disso? Decerto não foi tua arte.



\personagem{Tirésias}   Tu, pois tu me obrigaste a falar sem que eu quisesse.

\personagem{Édipo}   A falar o quê? Fala de novo, para que eu entenda melhor.

\personagem{Tirésias}   Testas-me com palavras?\NL{360}  Não compreendeste antes?

\personagem{Édipo}   Não a ponto de dizer que isso está conhecido. Declara de novo!

\personagem{Tirésias}   Afirmo que és o assassino do homem cujo assassino estás buscando.

\personagem{Édipo}   Não dirás impunemente essa calamidade duas vezes.

\personagem{Tirésias}   Devo então dizer uma outra, para que te enfureças ainda mais?

\personagem{Édipo}   O quanto desejares, visto que isso será dito em vão.

\personagem{Tirésias}   Afirmo que sem perceber convives do modo mais ignominioso com os que te
são mais caros e próximos, e não notas em que mal te encontras.

\personagem{Édipo}   Acaso pensas que dirás continuamente isso sem nenhum castigo?

\personagem{Tirésias}   Sim, se de fato a verdade tem algum poder.

\personagem{Édipo}   Mas tem, exceto para ti;\NL{370}  tu não o tens, visto que és cego dos
ouvidos, do intelecto e dos olhos.

\personagem{Tirésias}   E tu és um infeliz, ao lançar-me injúrias que contra ti todos estes, sem
exceção, logo lançarão.

\personagem{Édipo}   Tu te nutres apenas da noite, de modo que nunca poderias prejudicar nem
a mim, nem a outro que vê a luz.

\personagem{Tirésias}   Não é teu destino haver de cair pelas minhas mãos: pois basta Apolo, a
quem importa executar isso.

\personagem{Édipo}   Essas descobertas são de Creonte, ou de quem?

\personagem{Tirésias}   Creonte não é desgraça alguma para ti, mas tu mesmo.

\personagem{Édipo}   Ó opulência, realeza\NL{380} e arte que supera a arte em vida muito
cobiçada! Quanta inveja é guardada junto a vós se, por causa deste
governo, que a cidade me pôs nas mãos como algo dado, não pedido,
Creonte, o confiável, o amigo desde o princípio, rastejou até mim em
segredo e deseja dele banir-me, depois de à socapa enviar tal feiticeiro
urdidor de artifícios, um pedinte insidioso, cujos olhos estão fixos nos
ganhos, mas é cego em sua arte.

\NL{390} Pois vai, diz, como tu podes ser um adivinho digno de fé? Por que
não falaste a estes cidadãos algo que os libertasse, quando a
cadela\footnote{Nova referência à Esfinge, que proferia seus enigmas em
  forma de cantos.} aqui cantava rapsódias? Por certo, interpretar o
enigma não era tarefa do homem que viria a chegar, mas havia a
necessidade de um oráculo, e em relação a isso te mostraste sem qualquer
conhecimento advindo quer dos pássaros, quer de algum dos deuses. Mas
eu, Édipo, o que nada sabe,\footnote{Espécie de ironia dramática de
  segundo grau. Com esse sarcasmo, Édipo pensa estar sendo irônico,
  quando enuncia uma verdade de fato.} vim e a fiz cessar,
tendo encontrado a resposta por meio da inteligência, sem aprendê-la dos
pássaros. Sim, eu, que tu tentas banir, por pensares que estarás ao lado
do trono de\NL{400} Creonte. Penso, porém, que tu e aquele que tramou isso
chorareis ao expulsardes a maldição, e, se não parecesses um velho,
conhecerias, sofrendo, precisamente as mesmas traições que tens em
mente.

\personagem{Coro}   Tanto as palavras dele quanto as tuas, Édipo, parecem"-nos, quando as
comparamos, ditas com ira. Não precisamos disso, mas sim de investigar
como resolveremos do melhor modo o oráculo do deus.

\personagem{Tirésias}   Mesmo que exerças o poder real, ao menos minha resposta em iguais termos
\NL{410} deve ser igualada, pois também eu tenho esta força. Não vivo como
teu escravo, mas como o de Lóxias, de sorte que não estarei inscrito sob
a chefia de Creonte. Como me injuriaste com minha própria cegueira, digo
que tu, mesmo tendo olhos, não vês o mal em que estás, nem onde habitas,
nem com quem resides. Acaso sabes qual é tua origem? Não percebes que és
odioso aos teus -- tanto aos debaixo, quanto aos em cima da terra -- e o
duplo golpe da imprecação de teu pai e de tua mãe, imprecação de pés
terríveis, te impelirá desta terra, tendo tu agora um olhar direito,
mas, depois, obscuro.\NL{420} Qual não será o porto dos teus gritos? Que
Citéron em breve não estará em consonância com eles,\footnote{Nesse passo
  é seguido o texto dos manuscritos, mantendo a imagem naval que ecoará
  logo a seguir no texto, e não o texto de Lloyd-Jones que se vale de
  uma correção de Blaydes e Herwerden, o que resultaria em ``Que
  Hélicon, que Citéron em breve não estará em consonância com teus
  gritos''.} quando te tornares ciente das núpcias que fizeste navegar
ao ancoradouro inseguro de tua casa, depois de te deparares com uma boa
viagem? Não percebes a multidão dos outros males que te destruirão com
teus filhos. Assim, suja Creonte de lama, suja também a minha boca, pois
nenhum mortal jamais será mais desgraçadamente destroçado do que tu.

\personagem{Édipo}   Devo mesmo suportar\NL{430}  ouvir isso dele? Que te arruínes! Não irás de
novo embora, retirando-te depressa desta casa?

\personagem{Tirésias}   Se não tivesses chamado, eu não teria vindo.

\personagem{Édipo}   Eu não sabia de modo algum que irias dizer tolices, pois, do contrário,
dificilmente te teria conduzido à minha casa.

\personagem{Tirésias}   Somos esses tolos, como parece a ti, mas, aos teus pais, que te geraram,
éramos sensatos.

\personagem{Édipo}   Que pais? Espera! Que mortal me gerou?

\personagem{Tirésias}   Este dia te gerará e destruirá.

\personagem{Édipo}   Quão enigmático e indistinto é tudo o que dizes!

\personagem{Tirésias}   Não és excelente\NL{440}  em descobrir coisas assim?

\personagem{Édipo}   Insulta-me! Em tais insultos descobrirás minha grandeza.

\personagem{Tirésias}   Foi justamente esse fato que te destruiu.

\personagem{Édipo}   Mas, se salvei a cidade, não me importo.

\personagem{Tirésias}   Partirei então, e tu, menino, me leva!

\personagem{Édipo}   Sim, que leve! Já que aqui presente tu és um tormento em meu caminho. Se
te apressares, não poderás nos causar mais dor.

\personagem{Tirésias}   Partirei, depois de ter dito aquilo em razão do que vim, sem temer tua
face, pois\NL{450} não é possível que me destruas. Digo-te: esse homem,
que há muito investigas, fazendo ameaças e proclamações de busca sobre a
morte de Laio, ele está aqui, um estrangeiro, um imigrante, pelo que se
diz; mas a seguir se mostrará nascido em Tebas, e não se alegrará com
essa circunstância. Como um cego depois de ter visto e um mendigo em
lugar de homem rico, viajará para solo estrangeiro, apontando com o
cetro a terra diante de si. Ele mesmo se mostrará tanto irmão quanto
pai dos filhos com quem\NL{460} convive; da mulher de quem nasceu, filho e
esposo; e do pai, sócio do leito e assassino. Vai para dentro e pensa
nisso. Se me surpreenderes mentindo, diz que eu já não sei nada com
minha arte profética.

\saida{Sai \versal{Tirésias}.} 

\saida{\versal{Édipo} abandona o palco e vai para o palácio.}

\section{Primeiro estásimo}

\personagem{Coro} 

\entrada{1ª estr.}

\settowidth{\versewidth}{e, terríveis, seguem-no ao mesmo tempo}
%xxxxxxxxxxxxxxxxxxxxxxxxxxxxxxxxxxxxxxxxxxxxxxxxxxx
\begin{verse}[\versewidth]
Quem é aquele que a profética\\ 
rocha délfica cantava\\
ter cometido com mãos sanguinárias\\
o ato mais nefando dos nefandos?\\
É hora de mover em fuga\\
os pés com mais força\\
que os cavalos tempestuosos,\\
pois, armado com fogo e relâmpagos,\\
o filho de Zeus salta contra ele,\NL{470}\\
e, terríveis, seguem-no ao mesmo tempo\\
as Queres infalíveis.\\!
\end{verse}
  \entrada{1ª ant.}

\settowidth{\versewidth}{e, terríveis, seguem-no ao mesmo tempo}
%xxxxxxxxxxxxxxxxxxxxxxxxxxxxxxxxxxxxxxxxxxxxxxxxxxx
\begin{verse}[\versewidth]
Brilhou há pouco a mensagem,\\ 
revelando-se do nevado Parnaso,\\
para que todos rastreiem\\
o homem oculto.\\
Ele vaga sob bosques\\
selvagens, por cavernas\\
e pelas pedras, como um touro,\\
abandonado, mísero com pé mísero,\\
mantendo afastados os oráculos\NL{480}\\
do umbigo central da terra, mas eles,\\
sempre vivos, circunvoam-no.\\!
\end{verse} 
\entrada{2ª estr.}

\settowidth{\versewidth}{a fama de Édipo que se estende entre o povo,x}
%xxxxxxxxxxxxxxxxxxxxxxxxxxxxxxxxxxxxxxxxxxxxxxxxxxx
\begin{verse}[\versewidth]
Em terror, em terror agora me agita\\ 
o sábio intérprete das aves,\\
sem minha aprovação ou recusa,\\
e, perplexo, não sei o que dizer.\\
Alço voos de esperança sem visão\\
do presente nem do futuro.\\
Que injúria houve aos labdácidas\footnote{Ou seja, os descendentes de Lábdaco, ancestral de Laio e Édipo.}
\qb ou ao filho de Pólibo,\NL{490}\\
nem antes nem agora\\
eu mesmo jamais entendi,\\
para que então,\\
depois de pô-la à prova,\\
eu vá contra\\
a fama de Édipo que se estende entre o povo,\\
aliando-me aos labdácidas na vingança das\qb mortes obscuras.\\!
\end{verse}
  \entrada{2ª ant.}
\settowidth{\versewidth}{ele jamais trará sobre si a pecha da maldade.x}
%xxxxxxxxxxxxxxxxxxxxxxxxxxxxxxxxxxxxxxxxxxxxxxxxxxx
\begin{verse}[\versewidth]
Zeus e Apolo, por certo,\\ 
são sagazes e conhecedores do que é mortal;\\
mas, dentre os homens, não há juízo\\
verdadeiro que afirme que um adivinho\NL{500}\\
leva vantagem sobre mim;\\
e um homem pode sobrepujar\\
sabedoria por meio de sabedoria.\\
Eu jamais consentiria\\
com os que o censuram,\\
antes que visse essa palavra tornar-se certa;\\
pois a donzela alada\\
veio outrora manifesta\\
contra ele, e, com essa prova, viu-se\\
que era sábio e caro à cidade. Assim, vinda\qb do meu peito,\NL{510}\\
ele jamais trará sobre si a pecha da maldade.
\end{verse}

\section{Segundo episódio}

\entrada{Entra \versal{Creonte.}}


\personagem{Creonte}   Homens da cidade, ciente de que o rei Édipo me acusa com palavras
terríveis, aqui chego, incapaz de tolerá-las. Se nas circunstâncias
presentes ele pensa sofrer de minha parte, quer por atos, quer por
palavras, algo que leve a seu prejuízo, então já não\NL{520} desejo uma
vida duradoura, suportando esse rumor. Sim, o dano desse discurso não
leva a algo simples para mim, mas a uma questão maior, se eu tiver de
ser chamado não só de maldoso na cidade, mas de maldoso por ti e pelos
amigos.

\personagem{Coro}   Essa ofensa, com efeito, ocorreu, mas talvez tenha sido forçada mais
pela ira que pela inteligência dos pensamentos.

\personagem{Creonte}   Manifestou-se palavra segundo a qual o adivinho, persuadido pelos meus
conselhos, teria pronunciado discursos falsos?

\personagem{Coro}   Isso foi dito, mas sei que sem qualquer discernimento.

\personagem{Creonte}   Com olhos firmes e com firme peito essa acusação foi feita contra mim?

\personagem{Coro}    Não sei, não vejo\NL{530} o que fazem os poderosos; mas ele mesmo de sua
casa já avança para cá.

\entrada{Entra \versal{Édipo} abruptamente.}

\personagem{Édipo}   Ei, tu, como vieste aqui? Tens mesmo tamanha ousadia em tua face, a
ponto de vires à minha casa, quando és o manifesto assassino deste homem
aqui e o evidente saqueador do meu poder real? Vai, pelos deuses, diz!
Decidiste fazer isso por teres visto em mim alguma covardia ou
estultice? Ou que eu não reconheceria que este feito rastejava com
insídia contra mim e que, depois de perceber, dele não me defenderia?
\NL{540} Acaso não é tola a tua empresa: caçar o poder real -- algo que se
atinge com multidão e posses -- sem riqueza e sem amigos?

\personagem{Creonte}   Sabes o que deves fazer? Escuta por tua vez o que será dito em resposta
e, depois de entenderes, julga tu mesmo.

\personagem{Édipo}   Tu és hábil em falar; e eu, mau em compreender-te, pois te descobri
hostil e pesado comigo.

\personagem{Creonte}   Por isso mesmo, escuta primeiro agora o que direi.

\personagem{Édipo}   Por isso mesmo, não me declares que não és mau.

\personagem{Creonte}   Se de fato consideras que a arrogância é um bem à parte do intelecto,
não\NL{550} pensas certo.

\personagem{Édipo}   Se de fato consideras que não te submeterás a uma pena, mesmo tendo
feito mal a um homem da família, não pensas bem.

\personagem{Creonte}   Concedo que essas palavras sejam justas; mas me explica que sofrimento
tu afirmas ter sofrido.

\personagem{Édipo}   Tu me persuadiste ou não persuadiste de que eu deveria enviar alguém ao
venerando adivinho?

\personagem{Creonte}   E ainda agora me identifico com aquele conselho.

\personagem{Édipo}  Há quanto tempo em verdade Laio já\ldots{}  

\personagem{Creonte}   Realizou que feito? Não estou entendendo.

\personagem{Édipo}    Se esvaiu da\NL{560} vista por força de um assalto mortal?

\personagem{Creonte}   Um tempo longo e antigo teria de ser contado.

\personagem{Édipo}   Esse adivinho estava então em poder de sua arte?

\personagem{Creonte}   Era igualmente sábio, e honrado do mesmo modo.

\personagem{Édipo}   Ele fez naquele tempo alguma menção a mim?

\personagem{Creonte}   Não, em lugar algum, ao menos enquanto eu estava ao lado.

\personagem{Édipo}   Mas vós não mantivestes uma busca pelo matador?

\personagem{Creonte}   Mantivemos, como não? E nada ouvimos.

\personagem{Édipo}   Mas como então esse sábio não falou isso?

\personagem{Creonte}   Não sei. Costumo calar-me sobre o que não entendo.

\personagem{Édipo}   Isto ao menos tu sabes\NL{570}  e dirias se tivesses bom senso.

\personagem{Creonte}   Isto o quê? Se de fato eu tenho esse conhecimento, não o recusarei.

\personagem{Édipo}   Que, se ele não tivesse vindo em tua companhia, jamais teria falado do
meu assassínio contra Laio.

\personagem{Creonte}   Se ele está dizendo isso, tu mesmo sabes. E eu reputo justo aprender de
ti, como agora tu, de mim.

\personagem{Édipo}   Entende isto: por certo não serei flagrado como um assassino.

\personagem{Creonte}   Pois bem, estás casado com a minha irmã?

\personagem{Édipo}   Não há como negar o que me indagas.

\personagem{Creonte}   E governas a terra, conferindo igualmente a ela a mesma parte de poder?

\personagem{Édipo}   Tudo o que quer,\NL{580}  ela obtém de mim.

\personagem{Creonte}   Acaso não me igualo, como um terceiro, a vós dois?

\personagem{Édipo}   Mas é precisamente aí que te mostras um mau amigo.

\personagem{Creonte}   Não, se, como eu, ponderares contigo mesmo. Examina primeiro isto: se
pensas que alguém escolheria governar com medo mais do que com sono
tranquilo, se de fato há de possuir os mesmo poderes. Com efeito, nem eu
mesmo tenho por natureza o desejo de ser rei mais do que de exercer as
prerrogativas reais, nem qualquer outro que\NL{590} saiba ser temperante.
Agora trago sem medo tudo nas mãos por meio de ti; mas, se eu mesmo
governasse, muitas ações teria que executar contra minha vontade. Como
então a soberania seria para mim mais doce de possuir do que um comando
e um poder indolores? Ainda não me encontro tão enganado a ponto de
desejar algo diverso do que é belo e acompanhado de ganho. Agora me
alegro com todos, agora todos me saúdam, agora os que necessitam de ti
chamam-me de lado, pois é possível que tudo eles aqui encontrem. Como
então me apegaria àquelas coisas e abandonaria essas? {[}Não se\NL{600}
poderia tornar má a mente que pensa bem.{]}\footnote{Esse verso é
  retirado da edição de Wolff, e Lloyd-Jones o mantém entre colchetes na
  sua. O problema aqui é o da relação de seu significado com o dos
  versos que vêm em seguida.} Mas não sou amante dessa opinião, nem
suportaria estar com outro que assim agisse.

Como prova disso, vai a Pito e informa-te sobre os oráculos proferidos,
se os anunciei com clareza para ti. De resto, se me descobrires
deliberando algo em comum com o adivinho, não me mates, valendo-te de um
voto único, mas de um duplo, o meu e o teu; porém não me acuses por meio
de um julgamento incerto, feito apenas por ti\NL{610} mesmo. Não é justo
falsamente considerar bons os maus, nem maus os bons. {[}Afirmo que
banir um bom amigo é o mesmo que jogar fora a própria vida, a qual
sobretudo se ama.{]}\footnote{Lloyd-Jones sugere que se retirem esses
  dois versos do texto (611"-3), mantendo-os entre colchetes. Já van
  Deventer sugerira o apagamento até o verso 615.} Com o tempo sem
dúvida saberás disso, pois apenas o tempo revela o homem justo, ao passo
que o mau em um único dia podes conhecer.

\personagem{Coro}   Para quem toma o cuidado de não falhar, senhor, ele falou bem; pois os
rápidos em pensar são suscetíveis à queda.

\personagem{Édipo}   Quando um conspirador avança rápido em segredo, também eu por minha vez
\NL{620} devo rápido deliberar. Se permaneço tranquilo, seus atos estarão
executados, e os meus, perdidos.

\personagem{Creonte}   Que desejas afinal? Banir-me da terra?

\personagem{Édipo}   Nada disso; quero que tu morras, não que fujas.\footnote{Desde Bruhn, os editores apontam, a partir desse
  passo algumas lacunas no texto que afetam a própria distribuição das
  falas dos personagens e tornam o seu sentido um tanto confuso. Sigo a
  distribuição de Lloyd-Jones, que aponta uma lacuna de duas linhas
  nesta fala de Édipo e uma nova lacuna na próxima fala de Creonte.}

\personagem{Creonte}   Quando mostrares o que é a inveja\ldots{}

\personagem{Édipo}   Falas como se não houvesses de ceder nem de confiar em mim?

\personagem{Creonte}   Pois vejo que não pensas bem.

\personagem{Édipo}   Penso ao menos no meu interesse.

\personagem{Creonte}   Deves igualmente pensar também no meu.

\personagem{Édipo}   Mas tu és mau.

\personagem{Creonte}   E se não estiveres compreendendo nada?

\personagem{Édipo}   É preciso, contudo, que se deixe governar.

\personagem{Creonte}   Não, certamente, por um mau governante.

\personagem{Édipo}   Oh, cidade, cidade!

\personagem{Creonte}   Também\NL{630} a mim pertence a cidade, não apenas a ti.

\personagem{Coro}   Cessai com isso, soberanos. Aqui vejo Jocasta, vindo de casa em um
momento que vos é oportuno. Em companhia dela deveis bem dispor a
discórdia que ora se apresenta.

\entrada{Entra \versal{Jocasta.}}

\personagem{Jocasta}   Infelizes! Por que travais esta insensata guerra doméstica de palavras?
Não vos envergonhais de resolver males particulares enquanto a cidade
assim adoece? Vai para casa, e tu, Creonte, para tua morada, e não
leveis a uma grande dor o que não é nada.

\personagem{Creonte}   Minha irmã,\NL{640}  o teu esposo, Édipo, reputa justo cometer ações
terríveis contra mim, um ou outro dentre dois males: ou banir-me da
terra paterna, ou prender-me e matar-me.

\personagem{Édipo}   Confirmo, pois o surpreendi, minha mulher, agindo mal contra minha
pessoa por meio de maus artifícios.

\personagem{Creonte}   Que eu jamais prospere, mas que pereça em minha própria maldição, se
cometi algum dos crimes que me acusas de ter cometido.

\personagem{Jocasta}   Pelos deuses, Édipo, acredita nisso. Respeita antes de tudo esse
juramento aos deuses, depois a mim e a estes que estão ao teu lado.

\section{Primeiro \emph{kommós}}


\personagem{Coro} 
\entrada{Estr.}


\settowidth{\versewidth}{Eu suplico, senhor, consente nissox}
%xxxxxxxxxxxxxxxxxxxxxxxxxxxxxxxxxxxxxxxxxxxxxxxxxxx
\begin{verse}[\versewidth]
Eu suplico, senhor, consente nisso\\ 
com tua vontade e pensamento.\NL{650}
\end{verse}

\personagem{Édipo}   O que queres afinal que eu te conceda?


\personagem{Coro} 

\settowidth{\versewidth}{mas agora é grande em sua jura. Respeita-o!x}
%xxxxxxxxxxxxxxxxxxxxxxxxxxxxxxxxxxxxxxxxxxxxxxxxxxx
\begin{verse}[\versewidth]
Ele antes não era néscio,\\
mas agora é grande em sua jura. Respeita-o!
\end{verse}

\personagem{Édipo}   Sabes o que estás pedindo?

\personagem{Coro}   Sei.

\personagem{Édipo}   Fala, pois! O que estás dizendo?


\personagem{Coro} 

\settowidth{\versewidth}{Que jamais deves atingir com obscura acusação,X}
%xxxxxxxxxxxxxxxxxxxxxxxxxxxxxxxxxxxxxxxxxxxxxxxxxxx
\begin{verse}[\versewidth]
Que jamais deves atingir com obscura acusação,\\
deixando sem palavra, um amigo sob juramento.
\end{verse}

\personagem{Édipo} 

\settowidth{\versewidth}{brilhante? Tenso, agito-me com medo em meu trêmulo}
%xxxxxxxxxxxxxxxxxxxxxxxxxxxxxxxxxxxxxxxxxxxxxxxxxxx
\begin{verse}[\versewidth]Fica então bem ciente de que, quando buscas isso,\\
estás buscando a minha morte ou meu exílio da terra.
\end{verse}


\personagem{Coro} 

\settowidth{\versewidth}{brilhante? Tenso, agito-me com medo em meu trê}
%xxxxxxxxxxxxxxxxxxxxxxxxxxxxxxxxxxxxxxxxxxxxxxxxxxx
\begin{verse}[\versewidth]Não! Pelo deus à frente de todos os deuses, o Sol,\NL{660}\\
pois que eu pereça da morte mais extrema,\qb desamparado\\
de deuses e amigos, se mantenho esse pensamento.\\
Mas a terra, ao consumir-se, desgasta meu coração\\
desgraçado, se estes males, advindos de vós,\\
se ligarem aos meus males antigos.
\end{verse}

\personagem{Édipo\protect\footnote{Como se nota por sua própria configuração, esta
  fala de Édipo e as próximas três (de Creonte, Édipo e Creonte novamente)
  inserem-se entre a estrofe e a antístrofe do \emph{kommós.} Sobre o
  \emph{kommós}, ver Apêndice, p.\,\pageref{resumo}.}}

Que ele vá então, ainda que eu deva morrer completamente, ou que em
desonra\NL{670} eu seja afastado com violência desta terra. Apiedo-me da
tua boca, digna de compaixão, não da dele; e ele será odiado onde quer
que esteja.

\personagem{Creonte}   É evidente que cedes ainda cheio de ódio, e és rude quando ultrapassas o
limite da fúria. Naturezas deste tipo são, para si mesmas, com justiça,
as mais dolorosas de se suportar.

\personagem{Édipo}   Será que não vais me deixar e sair daqui?

\personagem{Creonte}   Irei, são e salvo entre estes, depois de ter-te encontrado em completa
ignorância.

\saida{Sai \versal{Creonte}.}


\personagem{Coro} 

\entrada{Ant.}

\settowidth{\versewidth}{brilhante? Tenso, agito-me com med}
%xxxxxxxxxxxxxxxxxxxxxxxxxxxxxxxxxxxxxxxxxxxxxxxxxxx
\begin{verse}[\versewidth]Senhora, por que hesitas\\ 
em levá-lo para dentro de casa?
\end{verse}

\personagem{Jocasta}   Eu o farei, quando compreender o que ocorreu.\NL{680}


\personagem{Coro} 

\settowidth{\versewidth}%
{da discussão, mas algo também injusto}
%xxxxxxxxxxxxxxxxxxxxxxxxxxxxxxxxxxxxxxxxxxxxxxxxxxx
\begin{verse}[\versewidth]
Uma suspeita sem certeza surgiu\\
da discussão, mas algo também injusto\qb provoca feridas.
\end{verse}

\personagem{Jocasta}   Isso vem de ambos?

\personagem{Coro}   
\begin{verse}[\versewidth]Sim.\end{verse}

\personagem{Jocasta}   E qual foi a discussão?


\personagem{Coro} 

\settowidth{\versewidth}{brilhante? Tenso, agito-me com medo em meu trêmulo}
%1xxxxxxxxxxxxxxxxxxxxxxxxxxxxxxxxxxxxxxxxxxxxxxxxxxx
\begin{verse}[\versewidth]
Parece suficiente, suficiente a mim, que me preocupo\\
com nossa terra, que isso permaneça aqui,\qb onde cessou.
\end{verse}

\personagem{Édipo} 

\settowidth{\versewidth}{brilhante? Tenso, agito-me com medo em meu trêmulo}
%xxxxxxxxxxxxxxxxxxxxxxxxxxxxxxxxxxxxxxxxxxxxxxxxxxx
\begin{verse}[\versewidth]Vês aonde chegas tu, que és um homem de bom juízo,\\
ao desprezares meu interesse e embotares meu coração?
\end{verse}


\personagem{Coro} 

\settowidth{\versewidth}{brilhante? Tenso, agito-me com medo em meu trê}
%xxxxxxxxxxxxxxxxxxxxxxxxxxxxxxxxxxxxxxxxxxxxxxxxxxx
\begin{verse}[\versewidth]Senhor, eu já o disse não uma única vez,\NL{690}\\
mas fica sabendo que eu me mostraria um insensato,\\
sem acesso a qualquer sensatez, se abandonasse a ti,\\
que sopraste um vento favorável sobre a minha\qb terra querida\\
quando ela se agitava em tormentos.\\
Torna-te agora de novo um bom guia!
\end{verse}

\personagem{Jocasta}   Pelos deuses, senhor, explica também a mim contra que fato afinal
ergueste tamanha ira.

\personagem{Édipo}    Direi, pois te\NL{700} venero, mulher, mais do que a estes. Foi por causa
de Creonte, tais foram as tramas que ele armou contra mim.

\personagem{Jocasta}   Se vais dizer com clareza, em tua acusação, a causa da vossa discórdia,
fala então!

\personagem{Édipo}   Ele diz que sou o assassino de Laio.

\personagem{Jocasta}   Por saber ele mesmo disso, ou por ter sido instruído por outro?

\personagem{Édipo}   Nada disso. Para cá enviou um adivinho perverso, pois, ao menos no que
toca a si mesmo, ele libera totalmente sua boca.

\personagem{Jocasta}   Livra-te agora daquilo que falas, escuta-me e compreende que não há
coisa\NL{710} alguma mortal que participe da arte profética. Eu te
mostrarei em poucas palavras uma prova disso.

Um oráculo veio outrora a Laio -- não direi que do próprio Febo, mas de
seus serviçais -- profetizando que ele estaria destinado a morrer pelas
mãos do filho que viesse a nascer de mim e dele. Certo dia, como se diz,
ladrões estrangeiros o assassinam em uma encruzilhada de três vias. E
três dias não haviam se passado desde o nascimento do filho, quando ele
o atou nas articulações dos pés e o lançou em uma montanha\NL{720}
inacessível por meio das mãos de outros. Assim, Apolo não fez que ele se
tornasse o assassino do pai, nem que Laio pelas mãos do filho padecesse
da morte terrível que ele temia.

As palavras proféticas divisaram esses acontecimentos, e com nenhum
deles deves te preocupar, pois o próprio deus facilmente revelará aquilo
em que ele encontre utilidade.

\personagem{Édipo}   Que divagação da alma e revolução dos pensamentos se apoderam de mim
agora, mulher, quando te escuto!

\personagem{Jocasta}   Voltado para que tipo de preocupação tu falas isso?

\personagem{Édipo}  Pensei ter ouvido\NL{730} de ti que Laio foi morto junto a uma
encruzilhada de três vias.

\personagem{Jocasta}   Foi isso o que se disse e ainda não deixou de ser dito.

\personagem{Édipo}   E onde é esse lugar em que aquele infortúnio se deu?

\personagem{Jocasta}   A terra se chama Fócida, e a estrada, cindida e partindo de Delfos e de
Dáulis, conduz ao mesmo local.

\personagem{Édipo}   E quanto tempo decorreu desde esse fato?

\personagem{Jocasta}   Isso se anunciou à cidade um pouco antes de te mostrares no comando
desta terra.

\personagem{Édipo}   Zeus! O que decidiste fazer comigo?

\personagem{Jocasta}   O que é isso que se impõe em teu peito, Édipo?

\personagem{Édipo}   Não me perguntes\NL{740}  ainda, mas diz que aspecto tinha Laio e em que
ponto de sua juventude ele estava.

\personagem{Jocasta}   Era moreno, e sua cabeça começa a florescer com fios brancos. Não se
afastava muito da tua aparência.

\personagem{Édipo}   Ai de mim, desgraçado! Parece que há pouco eu me lançava sem saber em
terríveis imprecações.

\personagem{Jocasta}   Que dizes? Apavoro-me ao olhar para ti, senhor.

\personagem{Édipo}   Nutro um terrível temor de que o adivinho estivesse vendo. Mostrarás
mais se esclareceres ainda um fato.

\personagem{Jocasta}   Pois bem, embora eu receie, ouvirei e direi o que perguntares.

\personagem{Édipo}   Ele ia em pequeno\NL{750}  número, ou com muitos homens em sua companhia,
como um líder?

\personagem{Jocasta}   Eram cinco ao todo, e entre eles havia um arauto. Um único carro
conduzia Laio.

\personagem{Édipo}   Ai, ai! Isso já está claro. Quem foi afinal que vos relatou essa
história, mulher?

\personagem{Jocasta}   Um servo, que foi o único a regressar depois de ter se salvado.

\personagem{Édipo}   Por acaso ele agora se encontra em casa?

\personagem{Jocasta}   Não, de modo algum, pois desde que veio de lá e viu que estavas no poder
e que\NL{760} Laio havia morrido, suplicou, tomando a minha mão, que eu o
enviasse aos campos e aos pastores de rebanhos, para que estivesse o
mais longe do alcance dos olhos da cidade. E eu o enviei, pois, como
escravo, ele merecia receber um favor ainda maior do que esse.

\personagem{Édipo}   Será que ele poderia vir de novo até nós com rapidez?

\personagem{Jocasta}   É possível, mas por que determinas isso?

\personagem{Édipo}   Temo, mulher, que eu tenha falado em demasia; por isso, desejo vê-lo.

\personagem{Jocasta}   Sim, ele\NL{770} virá; mas também eu, senhor, mereço compreender o que é
tão insuportável para ti.

\personagem{Édipo}   Como avanço em direção a tamanha expectativa, não serás privada disso,
pois a quem melhor do que a ti eu poderia falar, no momento em que passo
por tal infortúnio?

Meu pai era Pólibo de Corinto, e minha mãe, Mérope, de ascendência
dória. Eu era considerado o maior dentre os cidadãos de lá, até que este
fato sobreveio a mim; fato digno de espanto, mas indigno de um grande
zelo de minha parte.

\NL{780} Um homem encheu-se de vinho em um jantar e, quando já estava
embriagado, disse de mim que eu era um filho forjado de meu pai.
Irritado, detive-me com dificuldade durante aquele dia, mas, no
seguinte, aproximei-me de minha mãe e de meu pai e interroguei"-os. Eles
consideraram intolerável a ofensa daquele que soltou essa história, e eu
me alegrei com a reação dos dois; contudo, aquilo me incomodava
continuamente, pois se arrastava com toda força em mim. Escondido da
mãe e do pai, dirigi-me a Pito, e Febo enviou-me de volta sem a honra de
obter aquilo pelo que fui até\NL{790} lá, mas a mim, para minha
infelicidade, fez outras revelações lamentáveis e terríveis, ao dizer
que eu haveria de me unir a minha mãe, mostraria aos homens uma
descendência insuportável de se ver e seria o assassino do pai que me
gerou.

Ouvi essas palavras e, a partir de então, havendo de calcular pelos
astros a posição de Corinto, passei a fugir para onde eu jamais iria ver
cumpridas as injúrias dos meus funestos oráculos. Enquanto andava,
cheguei ao lugar em que afirmas que esse rei\NL{800} morreu. Eu te direi a
verdade, minha mulher. Quando em minha viagem eu estava próximo dessa
estrada tripla, ali mesmo um arauto e um homem tal como descreves,
dentro de um carro puxado por cavalos, encontraram-me. O guia e o
próprio homem mais velho tentaram à força me expulsar do caminho. Movido
pela raiva, golpeei o condutor do carro que ia me virando para fora da
estrada. O homem mais velho, quando viu isso, aguardou o momento em que
eu passava ao lado do carro e atingiu-me no meio\NL{810} da cabeça com seu
duplo aguilhão. Ele, entretanto, não pagou pena igual, mas foi
rapidamente ferido por este braço com um bastão e de imediato rolou do
meio do carro para fora, de costas. Matei todos. Se algo familiar a Laio
mantém qualquer relação com esse estrangeiro, quem poderia agora ser
mais infeliz do que este homem aqui? Que homem seria mais odioso aos
deuses, homem ao qual não é permitido que nenhum estrangeiro nem cidadão
deem acolhida em sua casa, nem lhe dirijam palavra, devendo eles
afastá-lo de seus lares? Quanto a isso, não foi nenhum outro senão eu
quem pôs\NL{820} sobre mim mesmo essas imprecações. Estou maculando o
leito do morto com minhas mãos, pelas quais precisamente ele morreu. Sou
mau? Não sou completamente impuro, se de fato devo fugir, e nesse
desterro não é possível que eu veja os meus nem que ponha os pés na
terra paterna, sob a condição de me juntar em núpcias a minha mãe e
assassinar meu pai, Pólibo, que me criou e gerou? Se alguém julgasse que
esses males partiram de uma cruel divindade contra este homem aqui, não
poderia estar falando a\NL{830} verdade? Que de modo algum, de modo algum,
ó sagrada imponência dos deuses, eu veja esse dia, mas que eu desapareça
aos olhos dos mortais antes de ver tal mancha da desgraça atingir a mim
mesmo.

\personagem{Coro}   Isso nos causa medo, senhor. Mas mantém a esperança, até que compreendas
tudo então por meio daquele que estava lá.

\personagem{Édipo}   Sim, minha esperança vai apenas ao ponto de aguardar pelo homem, o
pastor.

\personagem{Jocasta}   E, quando ele houver aparecido, qual será afinal teu desejo?

\personagem{Édipo}    Eu te explicarei:\NL{840} se ele for descoberto contando-te a mesma
história, poderei ter escapado do infortúnio.

\personagem{Jocasta}   Que história tão marcante ouviste de mim?

\personagem{Édipo}   Dizias que ele te contou que ladrões o assassinaram. Se ele então falar
ainda do mesmo número, eu não matei, pois um não pode ser o mesmo que
muitos. Se, por outro lado, ele se referir claramente a um único homem,
um viajante solitário, esse feito já se inclina em minha direção.

\personagem{Jocasta}   Fica sabendo que foi desse modo que sua palavra se revelou, e não é
possível\NL{850} que ele rejeite isso em nova ocasião, pois a cidade
ouviu, não somente eu. E mesmo que, de qualquer modo, ele se afaste do
seu discurso anterior, jamais revelará com justa correção\footnote{Ou
  seja, mesmo que o escravo diga que o crime foi cometido por uma única
  pessoa, a previsão do oráculo continuaria, segundo Jocasta,
  equivocada. Assim, ela volta a atacar os oráculos para novamente
  atingir a credibilidade de Tirésias.} o assassinato de Laio, o qual
Lóxias afirmou que deveria morrer pela ação do meu filho. Todavia, esse
infeliz jamais o matou, mas ele mesmo morreu antes. Assim, eu não
poderia olhar para um lado nem para outro por causa de um oráculo.

\personagem{Édipo}   É uma boa consideração, mas envia, contudo, alguém que mande para cá o
\NL{860} escravo, e não descuides disso.

\personagem{Jocasta}   Vou me apressar a enviá-lo. Mas vamos para casa, pois eu não faria nada
que não fosse caro a ti.

\saida{\versal{Édipo} e \versal{Jocasta} saem de cena.}

\section{Segundo estásimo}


\personagem{Coro} 
\entrada{1ª estr.}

\settowidth{\versewidth}{Que o destino esteja comigo enquanto eu portexx}
%xxxxxxxxxxxxxxxxxxxxxxxxxxxxxxxxxxxxxxxxxxxxxxxxxxx
\begin{verse}[\versewidth]
Que o destino esteja comigo enquanto eu porte\\ 
a pureza sagrada de todos os atos\\
e palavras para os quais as leis altaneiras\\
estão dispostas, geradas\\
no éter celeste. Delas o Olimpo\\
é o único pai, e a mortal\\
natureza dos homens\\
não as deu à luz, nem jamais o esquecimento\NL{870}\\
as fará repousar.\\
Grande nelas é o deus, e ele não envelhece.\\!
\end{verse}

  \entrada{1ª ant.}

% \settowidth{\versewidth}{brilhante? Tenso, agito-me com medo em meu trêmulo}
% %xxxxxxxxxxxxxxxxxxxxxxxxxxxxxxxxxxxxxxxxxxxxxxxxxxx
\begin{verse}[\versewidth]
A soberba produz o tirano; a soberba,\\ 
se em vão muito se farta\\
do que não convém nem é oportuno,\\
eleva-se às mais extremas cornijas\\
e arroja-se aos cumes abruptos da necessidade,\\
onde seu útil pé não lhe será\\
de utilidade. Quanto à luta\\
que é boa à cidade, peço a deus\NL{880}\\
que ele jamais a desfaça.\\
Não cesso jamais de ter o deus à minha frente.\\!
\end{verse} 
  \entrada{2ª estr.}

\settowidth{\versewidth}{brilhante? Tenso, agito-me com medo em meu trêm}
%xxxxxxxxxxxxxxxxxxxxxxxxxxxxxxxxxxxxxxxxxxxxxxxxxxx
\begin{verse}[\versewidth] 
Se alguém avança com insolência\\ 
em feitos ou palavras,\\
sem temor da Justiça\\
nem veneração às sedes das divindades,\\
que um nocivo destino o arrebate\\
por seu desventurado fausto,\\
se não há de ganhar seu ganho justamente\\
nem de manter-se afastado de feitos ímpios,\NL{890}\\
ou se há de como um tolo tocar no que é intocável.\\
Que homem nesse estado {[}terá ainda força\\
para afastar de sua vida as setas dos deuses?{]}\footnote{Tradução
baseada na solução dada por Dawe, que adota as emendas de Enger e
Hermann, para o texto dos manuscritos, considerado corrompido por
esses mesmos autores, por Lloyd- Jones e outros.}\\
Se tais ações são honrosas,\\
por que devo dançar em coros?\footnote{Isto é: por que devo honrar os
  deuses por meio de danças corais? Essa passagem é vista por alguns
  autores como um exemplo de metateatralidade, em que o Coro reflete
  sobre sua função, como que de fora da própria trama.}\\!
\end{verse} 

  \entrada{2ª ant.}

\settowidth{\versewidth}{brilhante? Tenso, agito-me com medo em meu trêmulo}
%xxxxxxxxxxxxxxxxxxxxxxxxxxxxxxxxxxxxxxxxxxxxxxxxxxx
\begin{verse}[\versewidth] 
Não mais em veneração irei\\ 
ao umbigo intocável da terra,\\
nem ao templo de Abas,\NL{900}\\
nem a Olímpia,\\
se essas profecias não se ajustarem aos fatos,\\
claras a todos os mortais.\\
Mas, ó Zeus soberano, rei supremo,\\
se com razão és assim chamado, que isso não escape\\
de ti nem de teu poder sempre imortal.\\
Já rejeitam os antigos oráculos\\
de Laio, que vão desvanecendo,\\
e em lugar algum está Apolo manifesto em honras.\\
O que é divino se esvai.\NL{910}
\end{verse}

\section{Terceiro episódio}

\entrada{\versal{Jocasta} entra carregando oferendas que dedicará à estátua de
Apolo sobre o palco.}

\personagem{Jocasta}   Senhores da terra, ocorreu-me a ideia de vir aos templos dos deuses com
estas guirlandas e estes incensos nas mãos, pois Édipo exalta em demasia
o seu ânimo com as mais variadas aflições e não julga, com um homem
sensato, os acontecimentos recentes com base nos antigos, mas se põe em
poder de quem fala, desde que lhe fale de terrores. Como não faço então
nenhum progresso com meus conselhos, venho a ti, Apolo lício,\NL{920} já
que és o mais próximo, como uma suplicante com estas oferendas votivas,
para que nos concedas um alívio purificador, pois agora todos tememos ao
vê-lo, como piloto de nossa nau, assim aturdido.

\entrada{Entra o mensageiro.}

\personagem{Mensageiro}   Eu poderia saber de vós, estrangeiros, onde é a casa do rei Édipo? Antes
de tudo, dizei se tendes ciência de onde ele está.

\personagem{Coro}   Esta é a sua residência, estrangeiro, e ele está lá dentro; e aqui está
sua mulher e mãe\footnote{Quando lido até esse ponto, o verso adquire um
  significado sinistro, que depois se desfaz (não obstante a ambiguidade
  de algum modo se mantenha).} dos seus filhos.

\personagem{Mensageiro}   Que ela sempre\NL{930} seja feliz e com gente feliz sempre esteja, visto
que ela é sua esposa perfeita.

\personagem{Jocasta}   Digo-te o mesmo, estrangeiro, pois disso és merecedor por tuas palavras
gentis. Mas me conta o que desejas com tua vinda, ou o que pretendes
anunciar.

\personagem{Mensageiro}   Boas notícias, senhora, a tua casa e a teu esposo.

\personagem{Jocasta}   Quais são elas? E tu vens da parte de quem?

\personagem{Mensageiro}   De Corinto. Com a mensagem que anunciarei logo poderás te alegrar --
como não? -- mas talvez com ela te aflijas.

\personagem{Jocasta}   O que há? Que duplo poder é esse que ela tem?

\personagem{Mensageiro}   Os habitantes\NL{940} da terra ístmica o instituirão como rei, é o que se
fala por lá.

\personagem{Jocasta}   O quê? O senhor Pólibo não mais está no poder?

\personagem{Mensageiro}   Não, não mais, pois a morte o tem em sua morada.

\personagem{Jocasta}   Como disseste? O pai de Édipo está morto?

\personagem{Mensageiro}   Se não estou falando a verdade, mereço morrer.

\personagem{Jocasta}   Escrava, vai com toda pressa e diz isso ao teu senhor. Ó oráculos dos
deuses, onde estais! Há muito, por temor de matar esse homem, Édipo
fugiu. E agora ele morreu por um acaso natural, e não pela ação daquele.

\entrada{Entra Édipo.}

\personagem{Édipo}   Cara Jocasta,\NL{950} mulher mais querida, por que me convocaste para vir
de casa para cá?

\personagem{Jocasta}   Escuta este homem, e depois de ouvi-lo observa aonde chegaram os
oráculos do deus.

\personagem{Édipo}   Quem é ele afinal e o que me diz?

\personagem{Jocasta}   Vem de Corinto, para anunciar que teu pai, Pólibo, não mais existe, ele
está morto.\footnote{O texto original, em grego, permite que se leia a
  primeira parte dessa fala da seguinte maneira: ``para anunciar
  que Pólibo não é mais teu pai''.}

\personagem{Édipo}   Que dizes, estrangeiro? Dá-me tu mesmo os sinais.

\personagem{Mensageiro}   Se devo primeiro anunciar com clareza, fica sabendo que ele trilhou o
caminho da morte.

\personagem{Édipo}   Por meio\NL{960} de insídias, ou pela intervenção de doença?

\personagem{Mensageiro}   Um pequeno solavanco deita corpos envelhecidos.

\personagem{Édipo}   O infeliz pereceu, como parece, por causa de doenças.

\personagem{Mensageiro}   E contando com um longo tempo de vida.

\personagem{Édipo}   Ai, ai! Por que, mulher, alguém olharia para o lar divinatório de Pito,
ou para os pássaros com seus alaridos ao alto, esses guias segundo os
quais eu haveria de matar meus pais? Ele oculta-se morto sob a terra, e
eu aqui estou sem ter tocado em uma arma qualquer, a menos que ele tenha
morrido por saudades de mim -- dessa forma ele teria\NL{970} sido morto por
minha causa. Quanto às profecias, porém, que se nos apresentaram, Pólibo
jaz no Hades, depois de levá-las consigo sem merecerem atenção alguma.

\personagem{Jocasta}   Eu não te predisse isso há muito tempo?

\personagem{Édipo}   Tu disseste, mas fui desviado pelo medo.

\personagem{Jocasta}   Não ponhas mais nada disso em teu coração.

\personagem{Édipo}   Mas como não devo temer o leito de minha mãe?

\personagem{Jocasta}   Por que um homem governado pelo acaso, para quem não há clara previsão
de nada, deveria temer? Muito melhor é viver a esmo, como se pode. Não
temas as núpcias\NL{980} com tua mãe, pois muitos homens também em sonhos
já se deitaram com a mãe. Leva a vida com mais facilidade aquele para
quem isso não é nada.

\personagem{Édipo}   Todas essas palavras estariam muito bem ditas por ti, se minha mãe não
estivesse viva. Mas agora, já que ela vive, é totalmente necessário que
eu receie, ainda que fales com propriedade.

\personagem{Jocasta}   Bem, os funerais de teu pai são ao menos uma grande fonte de
luz.\footnote{Literalmente: ``um grande olho''. A expressão pode ter o
  sentido de ``um grande conforto'', como sugerem alguns léxicos, mas é
  notável que, segundo a teoria ótica antiga, raios de luz emanariam dos
  olhos.}

\personagem{Édipo}   Sim, grande, compreendo; mas há um temor pela que está viva.

\personagem{Mensageiro}   Por que mulher vós tanto temeis?

\personagem{Édipo}   Por Mérope,\NL{990} com quem Pólibo habitava, ancião.

\personagem{Mensageiro}   E o que há com ela que vos leva assim a temer?

\personagem{Édipo}   Um terrível oráculo, estrangeiro, mandado por um deus.

\personagem{Mensageiro}   Ele pode ser dito? Ou não é lícito que outro o conheça?

\personagem{Édipo}   Sim, por certo! Lóxias disse uma vez que eu me deveria unir a minha
própria mãe e arrebatar com minhas mãos o sangue paterno. Por isso, há
muito passei a morar a uma grande distância de Corinto, de modo
próspero, é verdade; contudo, nada é mais doce que olhar para os olhos
dos pais.

\personagem{Mensageiro}   Acaso estavas\NL{1000} afastado daquela cidade por temeres isso?

\personagem{Édipo}   E por desejar, ancião, não ser o assassino de meu pai.

\personagem{Mensageiro}   Por que afinal não te livrei desse temor, senhor, já que vim de bom
coração?

\personagem{Édipo}   Por certo, de mim receberias uma merecida gratificação.

\personagem{Mensageiro}   Por certo, foi sobretudo por isso que vim, para auferir algum benefício
de ti, quando fores para teu lar.

\personagem{Édipo}   Mas jamais irei para junto de meus pais.

\personagem{Mensageiro}   Ó filho, é bem evidente que não sabes o que estás fazendo.

\personagem{Édipo}   Como assim, ancião? Explica-me isso, pelos deuses!

\personagem{Mensageiro}   Se por causa\NL{1010} disso evitas ir para o teu lar.

\personagem{Édipo}   Sim, por temer que Febo efetivamente se cumpra.

\personagem{Mensageiro}   E que tomes a mácula advinda dos pais?

\personagem{Édipo}   Isso mesmo, velho, é isso o que sempre me apavora.

\personagem{Mensageiro}   Acaso não percebes que tremes sem nenhuma razão?

\personagem{Édipo}   Como nenhuma, se sou filho desses pais?

\personagem{Mensageiro}   Porque Pólibo não mantinha contigo nenhum parentesco.

\personagem{Édipo}   Que disseste? Pólibo não me gerou?

\personagem{Mensageiro}  Não mais do que este homem aqui (\emph{Aponta para si.}), mas em igual
medida.

\personagem{Édipo}   Como meu genitor pode ser igual a quem nada é para mim?

\personagem{Mensageiro}   Mas ele\NL{1020} não te gerou, assim como eu também não.

\personagem{Édipo}   Então, por que ele me chamava de filho?

\personagem{Mensageiro}   Deves saber que ele outrora te recebeu das minhas mãos como um presente.

\personagem{Édipo}   E, mesmo tendo assim me recebido de mãos alheias, afeiçoou-se tanto a
mim?

\personagem{Mensageiro}   Sua anterior falta de filhos convenceu-o disso.

\personagem{Édipo}   Deste-me a ele depois de comprar-me ou achar-me?

\personagem{Mensageiro}   Após descobrir-te nos vales silvosos do Citéron.

\personagem{Édipo}   E por que viajavas por esses lugares?

\personagem{Mensageiro}   Lá, nas montanhas, eu guardava rebanhos.

\personagem{Édipo}   Eras um pastor e vagavas em serviço?

\personagem{Mensageiro}   Naquele\NL{1030} tempo, meu filho, fui teu salvador.

\personagem{Édipo}   Que aflição me acometia quando me tomaste nas mãos?

\personagem{Mensageiro}   As articulações dos teus pés poderiam testemunhar isso.

\personagem{Édipo}   Ai de mim! Por que te referes a esse antigo mal?

\personagem{Mensageiro}   Libertei-te quando tinhas as extremidades dos pés perfuradas.

\personagem{Édipo}   Um insulto terrível que obtive do berço.

\personagem{Mensageiro}   E assim, por causa desse infortúnio, foste chamado com o nome que
tens.\footnote{Édipo (\emph{Oidípous}) significa ``o de pés (\emph{poús})
  inchados (do verbo \emph{oidéō}, inchar-se, intumescer-se)''.}

\personagem{Édipo}   Por meu pai ou por minha mãe? Diz, em nome dos deuses!

\personagem{Mensageiro}   Não sei. Aquele que te deu a mim tem maior conhecimento disso do que eu.

\personagem{Édipo}   Recebeste-me então de outro e não me encontraste tu mesmo?

\personagem{Mensageiro}   Não,\NL{1040} um outro pastor deu-te a mim.

\personagem{Édipo}   Quem é ele? Sabes mostrar com palavras?

\personagem{Mensageiro}   Ele era chamado, suponho, um dos da casa de Laio.

\personagem{Édipo}   Do antigo rei desta terra?

\personagem{Mensageiro}   Sim, ele era um boieiro desse homem.

\personagem{Édipo}   Ele ainda está vivo, de modo que eu possa vê-lo?

\personagem{Mensageiro}   Vós, os habitantes da terra, deveríeis saber disso melhor do que
qualquer um.

\personagem{Édipo}   Há alguém, dentre vós aqui presentes, que conheça o boieiro a que ele se
refere,\NL{1050} quer o tenha visto nos campos, quer neste lugar? Indicai,
pois é hora de se descobrir isso!

\personagem{Coro}   Acho que não é ninguém mais do que aquele que se encontra nos campos e
que tu estavas ávido por ver; mas Jocasta, aqui, poderia falar melhor
disso.

\personagem{Édipo}   Minha mulher, sabes aquele que há pouco mandamos vir para cá? É esse o
homem de quem ele está falando?

\personagem{Jocasta}   Por que queres saber de quem ele falou? Não te preocupes, nem queiras te
lembrar disso, visto que foi dito em vão.

\personagem{Édipo}   Não haveria como eu não revelar minha raça depois de compreender tais
sinais.

\personagem{Jocasta}   Pelos\NL{1060} deuses! Se tens algum cuidado com tua própria vida, não
procures isso; meu flagelo é suficiente.

\personagem{Édipo}   Coragem! Pois nem se eu me mostrar um triplo escravo, nascido da
terceira geração de escravos, tu te revelarás vil.

\personagem{Jocasta}   Contudo, obedece-me, rogo-te, não faças isso!

\personagem{Édipo}   Não posso te obedecer se isso significa não entender tudo com clareza.

\personagem{Jocasta}   Mas é com prudência que falo o que é melhor para ti.

\personagem{Édipo}   Sim, esse teu ``melhor'' há muito vem me afligindo.

\personagem{Jocasta}   Ó desgraçado! Que jamais saibas quem és!

\personagem{Édipo}   Alguém\NL{1070} partirá e me trará o boieiro aqui? Quanto a ela, deixai-a
regozijar-se com sua rica estirpe.

\personagem{Jocasta}   Ah, ah, infeliz! É apenas disso que te posso chamar, e de nada mais
daqui em diante!

\saida{Sai \versal{Jocasta}.}

\personagem{Coro}   Por que, Édipo, tua mulher se foi sob esse ímpeto de dor selvagem? Temo
que males irrompam desse silêncio.

\personagem{Édipo}   Que irrompa o que se desejar; eu quero ver a minha origem, mesmo que
seja vil. Ela, talvez, por ser arrogante, como uma mulher, se envergonhe
de minha humilde\NL{1080} linhagem. Considero-me, porém, filho da Fortuna,
deusa doadora de bens, e não serei desonrado; pois nasci dessa mãe, e os
meses, meus parentes, determinaram minha grandeza e rebaixamento. Sendo
tal por natureza, jamais poderia me tornar outro, de modo a não
compreender a fundo o meu nascimento.

\entrada{\versal{Édipo} permanece no palco durante o terceiro estásimo.}

\section{Terceiro estásimo}


\personagem{Coro} 

\entrada{Estr.}

\settowidth{\versewidth}{não ignorarás que o plenilúnio de amanhãxx}
%xxxxxxxxxxxxxxxxxxxxxxxxxxxxxxxxxxxxxxxxxxxxxxxxxxx
\begin{verse}[\versewidth]Se um adivinho eu sou de fato\\
e sábio em meu juízo,\\
ó Citéron -- pelo Olimpo! --\\
não ignorarás que o plenilúnio de amanhã\\
te exalta como cidadão nativo\NL{1090}\\
e nutriz e mãe de Édipo,\\
e que és honrado em nossas danças\\
por portares dádivas\\
aos meus reis.\\
Ó Febo, com gritos invocado,\\
que isso seja do teu agrado.\\!
\end{verse} 
  \entrada{Ant.}
\settowidth{\versewidth}{que vaga pelas montanhas? Ou foi uma amante}
%xxxxxxxxxxxxxxxxxxxxxxxxxxxxxxxxxxxxxxxxxxxxxxxxxxx
\begin{verse}[\versewidth] 
Quem, filho, quem das deusas\\ 
de longa vida,\footnote{As ``deusas de longa vida'' são as ninfas.}\\
gerou-te então,\\
após se unir a Pã, teu pai,\NL{1100}\\ 
que vaga pelas montanhas? Ou foi uma amante\\
de Lóxias? A ele todos os altiplanos\qb de pastos agrestes são caros.\\
Ou foi o senhor de Cilene,\\
ou o deus báquico,\\
habitante dos cumes dos montes,\\
que te recebeu como um achado\\
de uma das ninfas de olhar vivaz,\\
com quem ele brinca amiúde?
\end{verse}


\section{Quarto episódio}

\personagem{Édipo}   Senhores,\NL{1110} se eu, que ainda não tive contato algum com ele, devo
arriscar uma conjectura, creio estar vendo o boieiro que há muito
buscamos, pois, na medida de sua velhice avançada, ele está em
consonância com este homem. Ademais, reconheço como meus os escravos que
o trazem. Mas talvez tu estejas à minha frente em conhecimento, já que
antes viste o boieiro.

\personagem{Coro}   Sim, reconheço-o, fica sabendo bem disso; pois ele era um dos de Laio e,
como pastor, mais fiel que qualquer outro.

\entrada{Entra o antigo servo de Tebas.}

\personagem{Édipo}   Indago-te\NL{1120} primeiro, estrangeiro de Corinto: é a este que te
referes?

\personagem{Mensageiro}   Precisamente este que estás vendo.

\personagem{Édipo}   Ei, tu, ancião, olha aqui e me diz o que eu te perguntar. Pertencias
outrora a Laio?

\personagem{Servo}   Sim, eu era um escravo não comprado, mas criado na casa.

\personagem{Édipo}   Com que tipo de tarefa, ou com que vida, te ocupavas?

\personagem{Servo}   A maior parte da vida eu seguia junto aos rebanhos.

\personagem{Édipo}   Em que lugares normalmente te abrigavas?

\personagem{Servo}   Havia o Citéron e havia as regiões próximas dali.

\personagem{Édipo}   Tens então ciência de ter conhecido este homem por lá?

\personagem{Servo}   O que ele fazia? De que homem estás falando?

\personagem{Édipo}   Este,\NL{1130} aqui presente; tiveste algum contato com ele?

\personagem{Servo}   Não a ponto de dizer assim rápido, de memória.

\personagem{Mensageiro}   Não é de espantar, senhor, mas eu o farei lembrar com clareza, tirando-o
da ignorância. Bem sei que ele tem pleno conhecimento de que, quando
habitávamos a região do Citéron, guardando ele um rebanho duplo, e eu,
um único, aproximei-me deste homem por três períodos inteiros, cada um
de um semestre, desde a primavera até a aparição de Arcturo; e no tempo
do inverno eu levava meu rebanho para o abrigo, e\NL{1140} ele, os dele
para os estábulos de Laio. É certo o que falo, ou o que falo não
ocorreu?

\personagem{Servo}   Falas a verdade, embora muito tempo tenha passado.

\personagem{Mensageiro}   Vai, diz então, sabes que naquele tempo me deste uma criança, para que
eu a criasse como minha própria cria?

\personagem{Servo}   O que há? Por que buscas saber isso?

\personagem{Mensageiro}   Este aqui, meu caro, é aquele que então era uma criança.

\personagem{Servo}   Que te arruínes! Não ficarás calado?

\personagem{Édipo}   Ah, ancião, não o castigues, pois tuas palavras necessitam mais de
castigo do que as dele.

\personagem{Servo}   Mas, ó melhor dos senhores, que erro cometo?\footnote{Inicia-se a partir daqui uma nova esticomitia,
  que vai até o verso 1170 (ver nota 13).}

\personagem{Édipo}   Não contas\NL{1150} a história da criança sobre a qual ele te indaga.

\personagem{Servo}   Ele fala sem nada saber e se esforça em vão.

\personagem{Édipo}   De bom grado não dirás, mas dirás sob suplício!

\personagem{Servo}   Pelos deuses, não me maltrates, sou um velho!

\personagem{Édipo}   Que alguém com toda pressa ponha suas mãos para trás!

\personagem{Servo}   Infeliz, em troca de quê? Que desejas compreender?

\personagem{Édipo}   Deste a este aqui a criança sobre a qual ele te indaga?

\personagem{Servo}   Dei, e quem me dera eu tivesse morrido naquele dia.

\personagem{Édipo}   Mas a isso chegarás, se não disseres o que deves.

\personagem{Servo}   Morro muito mais se eu contar.

\personagem{Édipo}   Este homem,\NL{1160} ao que parece, nos levará a gastar nosso tempo.

\personagem{Servo}   Não, de modo algum, mas há muito já disse que a dei.

\personagem{Édipo}   Depois de tomá-la de onde? De outra pessoa, ou ela era da tua família?

\personagem{Servo}   Não, não da minha; eu a recebi de alguém.

\personagem{Édipo}   De qual destes cidadãos? De que casa?

\personagem{Servo}   Pelos deuses, senhor, não procures saber mais.

\personagem{Édipo}   Estás morto, se eu te perguntar isso de novo.

\personagem{Servo}   Pois bem, foi um dos da casa de Laio.

\personagem{Édipo}   Um escravo, ou alguém nascido da família daquele?

\personagem{Servo}   Ai de mim! Ponho-me ao lado do próprio terror ao dizê"-lo.

\personagem{Édipo}   E eu ao\NL{1170} ouvi-lo! Contudo, devo ouvir.

\personagem{Servo}   Bem, chamavam a criança de filho dele, mas tua mulher, que se encontra
lá dentro, diria melhor como isso ocorreu.

\personagem{Édipo}   Foi ela quem deu a criança para ti.

\personagem{Servo}   Sim, senhor.

\personagem{Édipo}   Com que utilidade?

\personagem{Servo}   Para que eu a matasse.

\personagem{Édipo}   Depois de tê-la parido? Infeliz!

\personagem{Servo}   Sim, por medo de más profecias.

\personagem{Édipo}   Quais?

\personagem{Servo}   O que se dizia era que a criança mataria os pais.

\personagem{Édipo}   Por que então a entregaste a este velho?

\personagem{Servo}   Por piedade, senhor, porque eu pensava que ele haveria de levá-lo para
outra\NL{1180} terra, de onde ele mesmo era, mas ele a salvou para o mal
supremo; pois se és a própria pessoa que ele diz, fica sabendo que
nasceste em desgraça.

\personagem{Édipo}   Ai, ai! Tudo se pode tornar claro. Ó luz, que eu te olhe agora pela
última vez, eu que me mostro gerado por quem não devia, em consórcio com
quem não devia e assassino de quem não devia.

\saida{Saem \versal{Édipo}, o \versal{mensageiro} e o \versal{servo}.}

\section{Quarto estásimo}


\personagem{Coro} 

\entrada{1ª estr.}

\settowidth{\versewidth}{do que o bastante para parecerxx}
%xxxxxxxxxxxxxxxxxxxxxxxxxxxxxxxxxxxxxxxxxxxxxxxxxxx
\begin{verse}[\versewidth]Oh, gerações de mortais!\\ 
Como vos conto em vossa vida\\
por iguais ao nada!\\
Pois quem, que homem\\
porta mais felicidade\NL{1190}\\
do que o bastante para parecer\\
e, após ter parecido, decair?\\
Com teu nume como exemplo,\\
o teu, ó Édipo infeliz,\\
não julgo venturoso\\
nenhum dos mortais.\\!
\end{verse} 
  \entrada{1ª ant.}
\settowidth{\versewidth}{e foste honrado de modo supremo,xx}
%xxxxxxxxxxxxxxxxxxxxxxxxxxxxxxxxxxxxxxxxxxxxxxxxxxx
\begin{verse}[\versewidth] 
Tu, que com tuas flechas\\ 
excessivas conquistaste, não em tudo,\\
a feliz opulência -- ó Zeus! --\\
ao aniquilares a donzela\footnote{Nova alusão à Esfinge.} de garras\\
curvas, cantora de oráculos,\\
e te ergueste à minha terra\NL{1200}\\
como muro protetor de mortes;\\
por isso és chamado meu rei\\
e foste honrado de modo supremo,\\
como o senhor\\
da grandiosa Tebas.\\!
\end{verse} 
  \entrada{2ª estr.}
\settowidth{\versewidth}{Como? Como os campos por teu pai semeadosxx}
%xxxxxxxxxxxxxxxxxxxxxxxxxxxxxxxxxxxxxxxxxxxxxxxxxxx
\begin{verse}[\versewidth] 
Agora quem é mais infeliz de se ouvir?\\ 
Quem, por um revés da vida, convive\\
com desgraças entre penas mais selvagens?\\
Oh, célebre Édipo,\\
o mesmo grande porto\\
bastou para que tu,\\
como filho, e o pai,\\
nele repousásseis como noivos!\NL{1210}\\
Como? Como os campos por teu pai semeados\\
puderam, ó mísero, te suportar\\
por tanto tempo em silêncio?\\!
\end{verse} 
\entrada{2ª ant.}
\settowidth{\versewidth}{ele há muito condena as núpcias não nupciais,}
%xxxxxxxxxxxxxxxxxxxxxxxxxxxxxxxxxxxxxxxxxxxxxxxxxxx
\begin{verse}[\versewidth] 
O tempo, que tudo vê, descobriu-te em\qb ato involuntário;\\ 
ele há muito condena as núpcias não nupciais,\\
em que filhos geram e são gerados.\\
Oh, filho de Laio!\\
Que jamais, jamais,\\
eu tivesse te visto!\\
Como eu lamento,\\
ao verter de minha boca\\
o pranto desmedido! A bem dizer,\NL{1220}\\
graças a ti não só recobrei o fôlego,\\
mas também pus meus olhos a dormir.
\end{verse}


\section{Êxodo}

\entrada{Entra o \versal{segundo mensageiro}.}

\personagem{Segundo Mensageiro}   Homens desta terra, sempre os mais honrados: que feitos ouvireis, para
quais tereis olhado, e quanta dor suportareis, se de fato ainda lealmente
vos preocupais com a casa dos labdácidas! Creio que nem o Istro nem o
Fásis poderiam lavar e purificar esta morada, tamanhos os males que ela
esconde e que de imediato virão à luz, males\NL{1230} voluntários, não
involuntários. Dos sofrimentos o mais doloroso é o que se revela
autoinfligido.

\personagem{Coro}   O que antes sabíamos não deixa de merecer um grave lamento. Que tens a
dizer, além disso?

\personagem{Segundo Mensageiro}   As palavras mais sucintas a serem ditas e entendidas são estas: a divina
Jocasta está morta.

\personagem{Coro}   Oh, desgraçada! Mas qual foi a causa?

\personagem{Segundo Mensageiro}   Ela agiu por si mesma; e dos seus feitos, os mais dolorosos não estão ao
teu alcance, pois não se deram diante dos teus olhos. Contudo, pelo que
guardo em minha\NL{1240} memória, saberás dos sofrimentos daquela infeliz.

Depois que, movida pelo desvario, atravessou o vestíbulo, ela
imediatamente avançou para o leito nupcial, arrancando os cabelos com
ambas as mãos. E quando ingressou, após bater com força as portas,
passou a chamar Laio, já morto há muito, com a lembrança de quando, em
tempos passados, semearam o fruto por meio do qual ele morreria e
abandonaria essa mãe, genitora desgraçada, aos filhos dele mesmo.\NL{1250}
Ela lamentava sua cama, onde, duplamente infeliz, gerou do esposo um
esposo e filhos do filho. Não sei mais como depois disso ela morreu,
pois Édipo irrompeu aos gritos, e por isso não foi possível contemplar o
mal daquela, mas olhávamos para ele, que andava ao redor. Ele ia e
vinha, pedindo que lhe déssemos uma espada e sua esposa, que não era
esposa, perguntando onde ele encontraria o duplo campo materno, do qual
tanto ele quanto seus filhos brotaram. A ele, em sua fúria, alguma
divindade a mostrou, pois nenhum dos homens, que estávamos ali ao lado,
o fez. Com um grito terrível, como\NL{1260} que sob algum guia, ele
precipitou-se contra as portas duplas, inclinou-lhes os ferrolhos,
dobrando-os para fora de suas bases, e arrojou-se quarto adentro. Foi lá
que vimos a mulher suspensa, entrelaçada em laços trançados; e quando
ele, o infeliz, a avista, com um terrível grunhido desamarra a corda
suspensa, e, depois que ela estava deitada no chão, o que ali se passou
foi terrível de se ver. De suas roupas ele arrancou os alfinetes
dourados com que ela se vestira e, após erguê-los, feriu seus próprios
olhos,\NL{1270} anunciando que eles não veriam os sofrimentos que ele
padecia, nem os males que praticara, mas que pelo tempo restante
haveriam de ver na escuridão os que não deviam, e não conhecer os que
ele desejava que fossem conhecidos. A entoar tais cânticos ele muitas
vezes, e não uma única, voltava seus olhos para o alto e os golpeava. Ao
mesmo tempo as pupilas sangrentas molhavam sua face e não cessavam {[}de
escoar gotas de sangue, mas, de uma vez, uma negra chuva sanguínea se
derramava como granizo{]}.\footnote{Essa passagem é posta entre colchetes
  por Lloyd-Jones, seguindo West que sugere sua retirada do texto.}

\NL{1280} Esses males irromperam por causa de duas pessoas, e não de uma
apenas, e vêm misturados contra um homem e uma mulher. A antiga
opulência era antes justa opulência, mas agora, neste dia, é lamento,
desgraça, morte, vergonha e todos os males que se possam nomear: nenhum
está ausente.

\personagem{Coro}   E o desgraçado se encontra agora em algum repouso dos males?

\personagem{Segundo Mensageiro}   Aos gritos ele ordena que alguém destrave as portas e mostre a todos os
cadmeus o assassino do próprio pai e da própria mãe, proferindo palavras
ímpias, indizíveis por\NL{1290} mim, segundo as quais ele haveria de se
lançar para longe desta terra e não mais permanecer em sua morada sob a
maldição que ele mesmo imprecou. Da força, contudo, de algum guia ele
necessita, pois seu flagelo é maior que o suportável. Também a ti ele o
mostrará. As travas das portas se abrem, e logo verás um espetáculo tal
que com ele se compadeceria mesmo quem o odeia.


\section{Segundo \emph{kommós}}

\entrada{Entra \versal{Édipo}, depois de ter se cegado.}


\personagem{Coro} 

\settowidth{\versewidth}{para ti, tamanho é o temor que me provocas,}
%xxxxxxxxxxxxxxxxxxxxxxxxxxxxxxxxxxxxxxxxxxxxxxxxxxx
\begin{verse}[\versewidth] Ó sofrimento terrível aos olhos dos homens!\\
Ó mais terrível de todos com quantos\\
já me deparei! Que desvario, ó infeliz,\\
te acometeu? Que divindade, num salto\NL{1300}\\
maior que os mais longos, precipitou-se\\
contra teu desafortunado destino?\\
Ah, ah, mísero! Não sou capaz de olhar\\
para ti, tamanho é o temor que me provocas,\\
embora eu queira muito questionar,\\
muito saber e muito considerar.
\end{verse}

\personagem{Édipo} 

\settowidth{\versewidth}{para que terra sou levado? Por que aresx}
%xxxxxxxxxxxxxxxxxxxxxxxxxxxxxxxxxxxxxxxxxxxxxxxxxxx
\begin{verse}[\versewidth]
Ai, ai, ai, ai, mísero de mim, desgraçado,\\
para que terra sou levado? Por que ares\NL{1310}\\
voa minha voz assim carregada?\\
Oh, divindade, onde te precipitaste?
\end{verse}

\personagem{Coro}   No terrível, inaudível e invisível.

\personagem{Édipo} 
\entrada{1ª estr.}

\settowidth{\versewidth}{brilhante? Tenso, agito-me com medo em meu trêm}
%xxxxxxxxxxxxxxxxxxxxxxxxxxxxxxxxxxxxxxxxxxxxxxxxxxx
\begin{verse}[\versewidth]Oh, nuvem minha\\ 
de escuridão, abominável, nefanda em seu avanço,\\
indomável e por ventos desfavoráveis impelida!\\
Ai de mim!\\
Ai de mim de novo! Como a picada destes aguilhões\\
e a lembrança dos males em mim penetraram juntas!
\end{verse}


\personagem{Coro} 

\settowidth{\versewidth}{Não causa espanto algum que em tamanhas dores}
%xxxxxxxxxxxxxxxxxxxxxxxxxxxxxxxxxxxxxxxxxxxxxxxxxxx
\begin{verse}[\versewidth]
Não causa espanto algum que em tamanhas dores\\
te condoas e duplamente teus males lamentes.\NL{1320}
\end{verse}

\personagem{Édipo} 


\entrada{1ª ant.}

\settowidth{\versewidth}{ainda és meu companheiro constante, pois ainda}
%xxxxxxxxxxxxxxxxxxxxxxxxxxxxxxxxxxxxxxxxxxxxxxxxxxx
\begin{verse}[\versewidth]Oh, amigo,\\ 
ainda és meu companheiro constante, pois ainda\\
persistes em cuidar de mim, deste cego aqui.\\
Ah, ah!\\
Não deixo de te perceber, mas reconheço\qb com clareza\\
tua voz, embora eu esteja na escuridão.
\end{verse}


\personagem{Coro} 

\settowidth{\versewidth}{brilhante? Tenso, agito-me com medo em meu trêm}
%xxxxxxxxxxxxxxxxxxxxxxxxxxxxxxxxxxxxxxxxxxxxxxxxxxx
\begin{verse}[\versewidth]
Tu, que cometeste ações terríveis, como ousaste\\
assim destruir teus olhos? Que deus te exaltou?
\end{verse}

\personagem{Édipo} 
\entrada{2ª estr.}

\settowidth{\versewidth}{brilhante? Tenso, agito-me com medo em meu trêm}
%xxxxxxxxxxxxxxxxxxxxxxxxxxxxxxxxxxxxxxxxxxxxxxxxxxx
\begin{verse}[\versewidth]Foi Apolo, amigos, foi Apolo que executou\\ 
estes meus funestos, meus funestos sofrimentos;\NL{1330}\\
e ninguém, senão eu em minha desgraça,\\
os feriu com as próprias mãos.\\
Por que eu deveria ver,\\
se à minha visão nada havia de doce para ser visto?
\end{verse}

\personagem{Coro}   Foi bem assim como tu mesmo dizes.

\personagem{Édipo} 

\settowidth{\versewidth}{levai-me, amigos, levai esta grande ruína que sou,}
%xxxxxxxxxxxxxxxxxxxxxxxxxxxxxxxxxxxxxxxxxxxxxxxxxxx
\begin{verse}[\versewidth]O que afinal eu poderia ver\\
ou amar, ou que saudação\\
ainda ouvir com prazer, amigos?\NL{1340}\\
Levai-me o mais depressa para fora daqui,\\
levai-me, amigos, levai esta grande ruína que sou,\\
o mais amaldiçoado, e dentre os mortais\\
o mais odiado pelos deuses!
\end{verse}


\personagem{Coro} 

\settowidth{\versewidth}{brilhante? Tenso, agito-me com medo em meu trêm}
%xxxxxxxxxxxxxxxxxxxxxxxxxxxxxxxxxxxxxxxxxxxxxxxxxxx
\begin{verse}[\versewidth]Mísero, por tua mente e também por tua desgraça.\\
Quisera eu jamais te ter conhecido!
\end{verse}

\personagem{Édipo} 
\entrada{2ª estr.}

\settowidth{\versewidth}{da selvagem cadeia atada a meus pés, livrou-}
%xxxxxxxxxxxxxxxxxxxxxxxxxxxxxxxxxxxxxxxxxxxxxxxxxxx
\begin{verse}[\versewidth]
Que morra o pastor que me tirou\\ 
da selvagem cadeia atada a meus pés, livrou-me\NL{1350}\\
da morte e me salvou,\\
sem que me fizesse favor algum!\\
Pois se naquele momento tivesse eu morrido,\\
não seria um tamanho tormento aos amigos\qb nem a mim.
\end{verse}

\personagem{Coro}   Em meu desejo isso também se teria dado.

\personagem{Édipo} 

\settowidth{\versewidth}{e tenho filhos em comum com aquele de quem eu mesmo,}
%xxxxxxxxxxxxxxxxxxxxxxxxxxxxxxxxxxxxxxxxxxxxxxxxxxx
\begin{verse}[\versewidth]
Eu então para cá não teria vindo\\
como o assassino do pai nem pelos homens\\
seria chamado de noivo daquela de quem nasci.\\
Agora estou sem os deuses, sou fruto de pais\qb sacrílegos\NL{1360}\\
e tenho filhos em comum com aquele de\qb quem eu mesmo, desgraçado, nasci.\\
Se há um mal ainda mais grave do que o mal,\\
foi esse que Édipo obteve por sorte.
\end{verse}


\personagem{Coro} 

\settowidth{\versewidth}{Não sei como posso dizer que decidiste bem;}
%xxxxxxxxxxxxxxxxxxxxxxxxxxxxxxxxxxxxxxxxxxxxxxxxxxx
\begin{verse}[\versewidth]
Não sei como posso dizer que decidiste bem;\\
pois, se não mais existisses, estarias\qb melhor do que vivendo cego.
\end{verse}

\personagem{Édipo}   Não tentes\NL{1370} me explicar que isso não está feito da melhor maneira
nem me aconselhes mais, pois não sei com que olhos eu então olharia para
meu pai ao chegar ao Hades, nem tampouco para minha desgraçada mãe. Os
atos por mim cometidos contra ambos estão além do enforcamento.

Mas seria afinal desejável que eu olhasse para o semblante dos filhos,
tendo eles nascido como nasceram? Nunca, não ao menos com meus olhos!
Nem para a cidade, nem para seus muros, nem para as estátuas e templos
dos deuses; tudo isso de que eu,\NL{1380} para minha total desgraça,
depois de ter usufruído mais do que todos das maiores belezas de Tebas,
privei-me a mim mesmo ao determinar que todos afastassem o ímpio,
aquele que foi revelado pelos deuses impuro e da raça de Laio.

Uma vez que denunciei tal mácula pesando sobre mim, iria eu os ver com
olhos firmes? Jamais! Se ainda houvesse um bloqueio pelos ouvidos da
fonte da audição, eu não poderia deixar de fechar meu corpo infeliz,
para que, além de cego, eu nada\NL{1390} escutasse; pois é doce que o
pensamento habite longe dos males.

Oh, Citéron, por que me aceitaste? Por que não me mataste de imediato
quando me recebeste? Assim, aos homens eu jamais teria mostrado de onde
nasci. Ó Pólibo, Corinto e antiga casa supostamente paterna: de que modo
me criastes! Como uma beleza que por dentro se inflama com males! Pois
agora sou descoberto um homem mau, nascido de homens maus. Ó três
caminhos, vale encoberto, bosques e trilha que\NL{1400} se estreita em
vias triplas, vós, que bebestes o meu sangue, vertido pelas minhas mãos
do corpo do pai, acaso ainda vos lembrais dos atos que diante de vós
realizei e daqueles que, ao vir para cá, passei a praticar em seguida? Ó
núpcias, núpcias, vós me gerastes e, após essa geração, produzistes de
novo a mesma semente e revelastes pais que são irmãos, filhos de sangue
incestuoso, noivas que são esposas e mães e todos os atos que entre os
homens vêm a ser os mais torpes!

Mas, como não é belo falar do que tampouco é belo fazer, escondei-me, em
\NL{1410} nome dos deuses, com toda pressa possível, em algum lugar fora
daqui, ou matai-me, ou atirai-me ao mar, onde nunca mais me vereis!
Vinde! Dignai-vos tocar em um homem infeliz! Obedecei-me, não temais,
pois nenhum dos mortais é capaz de suportar os meus males senão eu.

\entrada{Entra Creonte.}

\personagem{Coro}   Mas Creonte aqui devidamente se apresenta para fazer e decidir o que
pedes, pois só ele resta como guardião da terra em teu lugar.

\personagem{Édipo}   Ai de mim,\NL{1420} que palavras dirigiremos a ele? Que justificada
garantia se mostrará em meu favor? Pois, em todos os meus atos
pretéritos dirigidos a ele, revelei-me um homem mau.

\personagem{Creonte}   Não vim zombar de ti, Édipo, nem te insultar por causa de alguma maldade
anterior. (\emph{Aos acompanhantes de Édipo}) Mas, se vós não mais
respeitais a raça dos mortais, tende ao menos, diante da chama do rei
Sol, que tudo alimenta, o pudor de mostrar assim descoberto tal
sacrílego, que nem a terra, nem a chuva sagrada, nem a luz acolherão.
Levai-o com toda pressa\NL{1430} para casa, pois é um ato de piedade que
apenas os da família vejam e apenas escutem os males familiares.

\personagem{Édipo}   Pelos deuses! Já que me arrancaste das más expectativas e vieste a mim,
como o melhor dos homens ao pior deles, faz o que te peço! É por ti que
falarei, não por mim.

\personagem{Creonte}   Que proveito assim insistes em obter de mim?

\personagem{Édipo}   Lança-me para fora desta terra o mais depressa possível, para onde
nenhum mortal poderá se dirigir a mim.

\personagem{Creonte}   Eu o teria feito, fica bem ciente disso, se não necessitasse primeiro
saber do deus como devo agir.

\personagem{Édipo}   Mas toda\NL{1440} a sua mensagem foi evidenciada: eu, o parricida, o
ímpio, devo morrer.

\personagem{Creonte}   Sim, isso foi dito; contudo, no estado de necessidade em que nos
encontramos, é melhor buscar saber o que deve ser feito.

\personagem{Édipo}   Portanto te informarás a respeito deste homem infeliz?

\personagem{Creonte}   Sim, e agora bem que poderias acreditar no deus.

\personagem{Édipo}   Também sobre ti ponho este encargo e te exorto: realiza, como tu mesmo
queiras, o funeral daquela que está em casa, pois de fato cumprirás
corretamente um dever em favor dos teus. Quanto a mim, porém, que esta
cidade paterna jamais se digne\NL{1450} de encontrar-me vivo, nela
morando; mas deixa-me habitar os montes, onde se situa essa minha
montanha, chamada Citéron, que meu pai e minha mãe, enquanto vivos,
fizeram ser meu decisivo túmulo, para que eu morresse pelas mãos deles
dois, que tentavam me matar.

Pelo tanto que sei, contudo, nem doença, nem qualquer outro infortúnio
poderiam destruir-me, pois eu jamais teria sido salvo da morte senão
para padecer um mal terrível. Mas que vá meu destino para onde ele for.
Com meus filhos, Creonte, não\NL{1460} é necessário que te preocupes: eles
são homens, de modo que, onde quer que estejam, jamais lhes faltarão
meios de vida. Mas das minhas duas filhas, infelizes e lamentáveis, à
parte das quais a mesa do meu alimento nunca foi posta, sempre as duas
compartilhando de tudo quanto eu tocasse e jamais restando sem este
homem aqui, cuida delas, por favor! E, mais do que tudo, deixa-me
tocá-las com as mãos e chorar meus males. Vai, senhor! Vai, ó tu, de
nobre nascimento! Se nelas eu pusesse as mãos,\NL{1470} eu poderia pensar
que as tenho, como quando eu enxergava. (\emph{Entram as filhas de
Édipo.}) Que estou dizendo? Pelos deuses, não estou ouvindo em algum
lugar minhas duas queridas vertendo lágrimas? Será que por piedade de
mim Creonte me enviou minhas duas filhas mais queridas? Estou certo?

\personagem{Creonte}   Estás. Fui eu que te proporcionei isso, por reconhecer a presente
satisfação, que antes também te possuía.

\personagem{Édipo}   Bem-aventurado sejas, e que, por tê-las trazido aqui, uma divindade te
proteja\NL{1480} mais do que protegeu a mim. Oh, filhas, onde afinal
estais? Vinde aqui, vinde a essas irmãs, as minhas mãos, cujo ofício fez
que os olhos outrora brilhantes do pai que vos engendrou passassem a ter
essa visão; ele que, minhas filhas, sem ver nem questionar se revelou
vosso pai a partir de quem ele mesmo foi gerado.

Choro por vós duas -- não tenho o poder de olhá-las -- ao pensar na
amargura de vossa vida restante. Como ambas podereis buscar dos homens
vossos meios de vida? A\NL{1490} que reuniões de cidadãos e a que festas
chegareis em retorno das quais não ireis para casa em pranto em vez de
alegres pelo que vistes? E quando chegardes ao momento das núpcias, quem
será ele? Quem assumirá o risco, filhas, de receber tais injúrias, que
serão igualmente um dano tanto a vossos pais quanto a vós mesmas? Que
mal está ausente? Vosso pai matou o pai, fecundou a mãe da qual ele
mesmo brotou, e vos obteve a partir da mesma mulher de quem ele nasceu.
Com essas injúrias sereis\NL{1500} repreendidas. Quem depois disso casará
convosco? Não há ninguém, minhas filhas, e é evidente que vós deveis
morrer estéreis e solteiras.

Mas, ó filho de Menécio, já que é o único pai restante destas duas, uma
vez que nós dois, que as geramos, estamos mortos; não as vejas, pai,
vagando como mendigas sem maridos, pois são parte da tua família, nem as
iguales aos meus males. Apieda-te delas, vendo-as desprovidas de tudo
nesta idade, exceto do quanto te cabe. Dá teu\NL{1510} consentimento,
nobre homem, com um toque de tua mão. A vós duas, minhas filhas, se já
tivésseis inteligência, muitos conselhos eu vos daria, mas agora fazei
esta prece por mim: a de viver onde a oportunidade permitir e encontrar
uma vida melhor do que a do pai que vos gerou.

\personagem{Creonte}   Já foste longe demais com teu pranto. Vai para dentro de casa!

\personagem{Édipo}   Devo obedecer, embora não seja nada agradável.

\personagem{Creonte}   Tudo é belo quando no momento oportuno.

\personagem{Édipo}   Sabes então em que condições irei?

\personagem{Creonte}   Tu dirás, e eu saberei quando te ouvir.

\personagem{Édipo}   Trata de me enviar para fora desta terra.

\personagem{Creonte}   Ao deus pertence o dom que me pedes.

\personagem{Édipo}   Mas sou o mais odioso aos deuses.

\personagem{Creonte}   Portanto, logo alcançarás isso.

\personagem{Édipo}   Dás então\NL{1520} tua confirmação?

\personagem{Creonte}   Não tenho o costume de dizer em vão o que não tenho em mente.

\personagem{Édipo}   Leva-me agora daqui!

\personagem{Creonte}   Anda, pois, e deixa tuas filhas irem.

\personagem{Édipo}   De modo nenhum as tires de mim!

\personagem{Creonte}   Não queiras mandar em tudo, pois é certo que as circunstâncias sob teu
comando não te seguiram por toda vida.

\saida{\versal{Édipo} e \versal{Creonte} saem de cena.}

\personagem{Coro}   Habitantes de Tebas, nossa terra paterna, vede! Este é Édipo, o que
desvendou os célebres enigmas e era o homem mais poderoso, para cuja
fortuna todo cidadão olhava com inveja. Vede em que mar de terrível
desgraça ele incorreu! Assim, seria preciso observar o dia derradeiro e
não julgar feliz nenhum mortal até que tenha\NL{1530} atravessado o termo
da vida sem sofrer dor alguma.

% \end{comment}
\endgroup % final da peça



